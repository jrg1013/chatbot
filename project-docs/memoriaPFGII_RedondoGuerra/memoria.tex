\documentclass[a4paper,12pt,twoside]{memoir}

% Castellano
\usepackage[spanish,es-tabla]{babel}
\selectlanguage{spanish}
\usepackage[utf8]{inputenc}
\usepackage[T1]{fontenc}
\usepackage{lmodern} % Scalable font
\usepackage{microtype}
\usepackage{placeins}

% Acrónimos
\usepackage[acronym]{glossaries}
\makenoidxglossaries

% Acrónimos
\newacronym{tfg}{TFG}{Trabajo de Fin de Grado}
\newacronym{llm}{LLM}{\textit{Large Language Models}}
\newacronym{rag}{RAG}{\textit{Retrieval-augmented Generation}}
\newacronym{ia}{IA}{Inteligencia Artificial}
\newacronym{ubu}{UBU}{Universidad de Burgos}
\newacronym{dnn}{DNN}{\textit{Deep Neural Networks}}
\newacronym{gpt}{GPT}{\textit{Generative Pre-trained Transformer}}
\newacronym{pln}{PLN}{Procesamiento del Lenguaje Natural}
\newacronym{mit}{MIT}{\textit{Massachusetts Institute of Technology}}
\newacronym{svm}{SVM}{Máquinas de soporte vectorial}
\newacronym{crf}{CRF}{Cadenas de Markov Condicionales}
\newacronym{cbow}{CBOW}{\textit{Continuous Bag of Words}}
\newacronym{api}{API}{\textit{Application Programming Interface}}
\newacronym{rlhf}{RLHF}{\textit{Reinforcement Learning from Human Feedback}}
\newacronym{cot}{CoT}{\textit{Chain-of-thought}}

\RequirePackage{booktabs}
\RequirePackage[table]{xcolor}
\RequirePackage{xtab}
\RequirePackage{multirow}

% Links
\PassOptionsToPackage{hyphens}{url}\usepackage[colorlinks]{hyperref}
\hypersetup{
	allcolors = {red}
}

% Ecuaciones
\usepackage{amsmath}

% Rutas de fichero / paquete
\newcommand{\ruta}[1]{{\sffamily #1}}

% Párrafos
\nonzeroparskip

% Huérfanas y viudas
\widowpenalty100000
\clubpenalty100000

% Imágenes

% Comando para insertar una imagen en un lugar concreto.
% Los parámetros son:
% 1 --> Ruta absoluta/relativa de la figura
% 2 --> Texto a pie de figura
% 3 --> Tamaño en tanto por uno relativo al ancho de página
\usepackage{graphicx}
\newcommand{\imagen}[3]{
	\begin{figure}[!h]
		\centering
		\includegraphics[width=#3\textwidth]{#1}
		\caption{#2}\label{fig:#1}
	\end{figure}
	\FloatBarrier
}

% Comando para insertar una imagen sin posición.
% Los parámetros son:
% 1 --> Ruta absoluta/relativa de la figura
% 2 --> Texto a pie de figura
% 3 --> Tamaño en tanto por uno relativo al ancho de página
\newcommand{\imagenflotante}[3]{
	\begin{figure}
		\centering
		\includegraphics[width=#3\textwidth]{#1}
		\caption{#2}\label{fig:#1}
	\end{figure}
}

% El comando \figura nos permite insertar figuras comodamente, y utilizando
% siempre el mismo formato. Los parametros son:
% 1 --> Porcentaje del ancho de página que ocupará la figura (de 0 a 1)
% 2 --> Fichero de la imagen
% 3 --> Texto a pie de imagen
% 4 --> Etiqueta (label) para referencias
% 5 --> Opciones que queramos pasarle al \includegraphics
% 6 --> Opciones de posicionamiento a pasarle a \begin{figure}
\newcommand{\figuraConPosicion}[6]{%
  \setlength{\anchoFloat}{#1\textwidth}%
  \addtolength{\anchoFloat}{-4\fboxsep}%
  \setlength{\anchoFigura}{\anchoFloat}%
  \begin{figure}[#6]
    \begin{center}%
      \Ovalbox{%
        \begin{minipage}{\anchoFloat}%
          \begin{center}%
            \includegraphics[width=\anchoFigura,#5]{#2}%
            \caption{#3}%
            \label{#4}%
          \end{center}%
        \end{minipage}
      }%
    \end{center}%
  \end{figure}%
}

%
% Comando para incluir imágenes en formato apaisado (sin marco).
\newcommand{\figuraApaisadaSinMarco}[5]{%
  \begin{figure}%
    \begin{center}%
    \includegraphics[angle=90,height=#1\textheight,#5]{#2}%
    \caption{#3}%
    \label{#4}%
    \end{center}%
  \end{figure}%
}
% Para las tablas
\newcommand{\otoprule}{\midrule [\heavyrulewidth]}
%
% Nuevo comando para tablas pequeñas (menos de una página).
\newcommand{\tablaSmall}[5]{%
 \begin{table}
  \begin{center}
   \rowcolors {2}{gray!35}{}
   \begin{tabular}{#2}
    \toprule
    #4
    \otoprule
    #5
    \bottomrule
   \end{tabular}
   \caption{#1}
   \label{tabla:#3}
  \end{center}
 \end{table}
}

%
% Nuevo comando para tablas pequeñas (menos de una página).
\newcommand{\tablaSmallSinColores}[5]{%
 \begin{table}[H]
  \begin{center}
   \begin{tabular}{#2}
    \toprule
    #4
    \otoprule
    #5
    \bottomrule
   \end{tabular}
   \caption{#1}
   \label{tabla:#3}
  \end{center}
 \end{table}
}

\newcommand{\tablaApaisadaSmall}[5]{%
\begin{landscape}
  \begin{table}
   \begin{center}
    \rowcolors {2}{gray!35}{}
    \begin{tabular}{#2}
     \toprule
     #4
     \otoprule
     #5
     \bottomrule
    \end{tabular}
    \caption{#1}
    \label{tabla:#3}
   \end{center}
  \end{table}
\end{landscape}
}

%
% Nuevo comando para tablas grandes con cabecera y filas alternas coloreadas en gris.
\newcommand{\tabla}[6]{%
  \begin{center}
    \tablefirsthead{
      \toprule
      #5
      \otoprule
    }
    \tablehead{
      \multicolumn{#3}{l}{\small\sl continúa desde la página anterior}\\
      \toprule
      #5
      \otoprule
    }
    \tabletail{
      \hline
      \multicolumn{#3}{r}{\small\sl continúa en la página siguiente}\\
    }
    \tablelasttail{
      \hline
    }
    \bottomcaption{#1}
    \rowcolors {2}{gray!35}{}
    \begin{xtabular}{#2}
      #6
      \bottomrule
    \end{xtabular}
    \label{tabla:#4}
  \end{center}
}

%
% Nuevo comando para tablas grandes con cabecera.
\newcommand{\tablaSinColores}[6]{%
  \begin{center}
    \tablefirsthead{
      \toprule
      #5
      \otoprule
    }
    \tablehead{
      \multicolumn{#3}{l}{\small\sl continúa desde la página anterior}\\
      \toprule
      #5
      \otoprule
    }
    \tabletail{
      \hline
      \multicolumn{#3}{r}{\small\sl continúa en la página siguiente}\\
    }
    \tablelasttail{
      \hline
    }
    \bottomcaption{#1}
    \begin{xtabular}{#2}
      #6
      \bottomrule
    \end{xtabular}
    \label{tabla:#4}
  \end{center}
}

%
% Nuevo comando para tablas grandes sin cabecera.
\newcommand{\tablaSinCabecera}[5]{%
  \begin{center}
    \tablefirsthead{
      \toprule
    }
    \tablehead{
      \multicolumn{#3}{l}{\small\sl continúa desde la página anterior}\\
      \hline
    }
    \tabletail{
      \hline
      \multicolumn{#3}{r}{\small\sl continúa en la página siguiente}\\
    }
    \tablelasttail{
      \hline
    }
    \bottomcaption{#1}
  \begin{xtabular}{#2}
    #5
   \bottomrule
  \end{xtabular}
  \label{tabla:#4}
  \end{center}
}



\definecolor{cgoLight}{HTML}{EEEEEE}
\definecolor{cgoExtralight}{HTML}{FFFFFF}

%
% Nuevo comando para tablas grandes sin cabecera.
\newcommand{\tablaSinCabeceraConBandas}[5]{%
  \begin{center}
    \tablefirsthead{
      \toprule
    }
    \tablehead{
      \multicolumn{#3}{l}{\small\sl continúa desde la página anterior}\\
      \hline
    }
    \tabletail{
      \hline
      \multicolumn{#3}{r}{\small\sl continúa en la página siguiente}\\
    }
    \tablelasttail{
      \hline
    }
    \bottomcaption{#1}
    \rowcolors[]{1}{cgoExtralight}{cgoLight}

  \begin{xtabular}{#2}
    #5
   \bottomrule
  \end{xtabular}
  \label{tabla:#4}
  \end{center}
}



\graphicspath{ {./img/} }

% Capítulos
\chapterstyle{bianchi}
\newcommand{\capitulo}[2]{
	\setcounter{chapter}{#1}
	\setcounter{section}{0}
	\setcounter{figure}{0}
	\setcounter{table}{0}
	\chapter*{\thechapter.\enskip #2}
	\addcontentsline{toc}{chapter}{\thechapter.\enskip #2}
	\markboth{#2}{#2}
}

% Apéndices
\renewcommand{\appendixname}{Apéndice}
\renewcommand*\cftappendixname{\appendixname}

\newcommand{\apendice}[1]{
	%\renewcommand{\thechapter}{A}
	\chapter{#1}
}

\renewcommand*\cftappendixname{\appendixname\ }

% Formato de portada
\makeatletter
\usepackage{xcolor}
\newcommand{\tutor}[1]{\def\@tutor{#1}}
\newcommand{\course}[1]{\def\@course{#1}}
\definecolor{cpardoBox}{HTML}{E6E6FF}
\def\maketitle{
  \null
  \thispagestyle{empty}
  % Cabecera ----------------
\noindent\includegraphics[width=\textwidth]{cabecera}\vspace{1cm}%
  \vfill
  % Título proyecto y escudo informática ----------------
  \colorbox{cpardoBox}{%
    \begin{minipage}{.8\textwidth}
      \vspace{.5cm}\Large
      \begin{center}
      \textbf{TFG del Grado en Ingeniería Informática}\vspace{.6cm}\\
      \textbf{\LARGE\@title{}}
      \end{center}
      \vspace{.2cm}
    \end{minipage}

  }%
  \hfill\begin{minipage}{.20\textwidth}
    \includegraphics[width=\textwidth]{escudoInfor}
  \end{minipage}
  \vfill
  % Datos de alumno, curso y tutores ------------------
  \begin{center}%
  {%
    \noindent\LARGE
    Presentado por \@author{}\\ 
    en Universidad de Burgos --- \@date{}\\
    Tutores: \@tutor{}\\
  }%
  \end{center}%
  \null
  \cleardoublepage
  }
\makeatother

\newcommand{\nombre}{José María Redondo Guerra} %%% cambio de comando

% Datos de portada
\title{Chatbot basado en \textit{Large Language Models} y técnicas de \textit{Retrieval-augmented Generation} para asistente de dudas sobre la realización del Trabajo Fin de Grado.}
\author{\nombre}
\tutor{José Ignacio Santos Martín, Carlos López Nozal}
\date{\today}

\begin{document}

\maketitle


\newpage\null\thispagestyle{empty}\newpage


%%%%%%%%%%%%%%%%%%%%%%%%%%%%%%%%%%%%%%%%%%%%%%%%%%%%%%%%%%%%%%%%%%%%%%%%%%%%%%%%%%%%%%%%
\thispagestyle{empty}


\noindent\includegraphics[width=\textwidth]{cabecera}\vspace{1cm}

\noindent D. José Ignacio Santos Martín, profesor del departamento de Ingeniería de Organización, área de Organización de Empresas y D. Carlos López Nozal, profesor del departamento Ingeniería Informática, área de Lenguajes y Sistemas Informáticos.

\noindent Exponen:

\noindent Que el alumno D. \nombre, con DNI 44906147G, ha realizado el Trabajo final de Grado en Ingeniería Informática titulado ``Chatbot basado en \textit{Large Language Models} y técnicas de \textit{Retrieval-augmented Generation} para asistente de dudas sobre la realización del Trabajo Fin de Grado''. 

\noindent Y que dicho trabajo ha sido realizado por el alumno bajo la dirección de los que suscriben, en virtud de lo cual se autoriza su presentación y defensa.

\begin{center} %\large
En Burgos, {\large \today}
\end{center}

\vfill\vfill\vfill

% Author and supervisor
\begin{minipage}{0.45\textwidth}
\begin{flushleft} %\large
Vº. Bº. del Tutor:\\[2cm]
D. José Ignacio Santos Martín
\end{flushleft}
\end{minipage}
\hfill
\begin{minipage}{0.45\textwidth}
\begin{flushleft} %\large
Vº. Bº. del co-tutor:\\[2cm]
D. Carlos López Nozal co-tutor
\end{flushleft}
\end{minipage}
\hfill

\vfill

% para casos con solo un tutor comentar lo anterior
% y descomentar lo siguiente
%Vº. Bº. del Tutor:\\[2cm]
%D. nombre tutor


\newpage\null\thispagestyle{empty}\newpage


\frontmatter

% Abstract en castellano
\renewcommand*\abstractname{Resumen}
\begin{abstract}

Este proyecto se centra en el desarrollo de un chatbot avanzado diseñado para proporcionar asistencia a estudiantes de Ingeniería Informática que se encuentran inmersos en la realización de sus \acrfull{tfg}. El chatbot se basa en \acrfull{llm}, como ChatGPT, y emplea técnicas de \acrfull{rag} para enriquecer su capacidad de respuesta y acceso a información relevante.

Este proyecto se caracteriza por su doble enfoque: en primer lugar, un análisis tecnológico exhaustivo de las distintas soluciones de implementación del chatbot utilizando LLM basados en \acrfull{ia} generativa; en segundo lugar, su aplicación en un entorno específico, en este caso, la asistencia a estudiantes en la realización de sus \acrshort{tfg} en la Universidad de Burgos. Se prioriza la evaluación tecnológica y el análisis crítico de las soluciones de implementación del chatbot, junto con la comparación de su desempeño con chatbots preexistentes.

El proyecto busca no solo explorar las capacidades de los \acrshort{llm} y las técnicas de \acrshort{rag}, sino también evaluar cómo estas tecnologías pueden tener un impacto significativo en un contexto concreto.


\end{abstract}

\renewcommand*\abstractname{Descriptores}
\begin{abstract}
\textit{Chatbot}, \textit{Large Language Models(LLM)}, \textit{Retrieval-augmented Generation(RAG)}, Procesamiento de lenguaje natural(NLP), Inteligencia artificial, Aprendizaje automático, \textit{Embeddings}, Base de datos vectorial,\ldots
\end{abstract}

\clearpage

% Abstract en inglés
\renewcommand*\abstractname{Abstract}
\begin{abstract}
This project focuses on the development of an advanced chatbot designed to provide assistance to Computer Engineering students who are immersed in the completion of their ``Final Thesis'' (\acrshort{tfg}). The chatbot is based on \acrfull{llm}, such as ChatGPT, and uses \acrfull{rag} techniques to improve its responsiveness and access to relevant information.

This project is characterized by its twofold approach: first, an exhaustive technological analysis of the different chatbot implementation solutions using \acrshort{llm} based on ``Generative Artificial Intelligence'' (AI); second, its application in a specific environment, in this case, the assistance to students in the realization of their \acrshort{tfg} at the University of Burgos. Priority is given to the technological evaluation and critical analysis of the chatbot implementation solutions, together with the comparison of its performance with pre-existing chatbots.

The project seeks not only to explore the capabilities of \acrshort{llm} and \acrshort{rag} techniques, but also to evaluate how these technologies can have a significant impact in a specific context.
\end{abstract}

\renewcommand*\abstractname{Keywords}
\begin{abstract}
Chatbot, Large Language Models(LLM), Retrieval-augmented Generation(RAG), Natural Language Processing(NLP), Artificial Intelligence, Machine Learning, Embeddings, Vector Database,\ldots
\end{abstract}

\clearpage

% Indices
\tableofcontents

\clearpage

\listoffigures

\clearpage

\listoftables
\clearpage

\mainmatter
\capitulo{1}{Introducción}

Como conclusión del grado en Ingeniería Informática, es necesaria la realización de un Trabajo de Fin de Grado (TFG) que marca el cierre de una etapa de formación. Sin embargo, este proyecto no está exento de desafíos, y los estudiantes a menudo se enfrentan a una multitud de preguntas y obstáculos, desde la selección del tema adecuado hasta la presentación final del proyecto. La complejidad de este proceso, combinada con la necesidad de optimizar los tiempos y la orientación, ha dado lugar a una necesidad de herramientas y recursos que asistan a los estudiantes en esta parte de su etapa académica.

En respuesta a esta demanda, este proyecto se centra en el desarrollo de un asistente virtual innovador: un chatbot basado en Large Language Models (LLM) que tiene como objetivo proporcionar respuestas precisas y adaptadas y orientación en tiempo real a los estudiantes que comienzan a realizar el TFG en Ingeniería Informática.

La base de este chatbot reside en su capacidad para comprender y procesar preguntas complejas, así como en su habilidad para generar respuestas coherentes y relevantes. Para mejorar aún más su eficacia y precisión, se implementaran técnicas de "Retrieval-augmented Generation" (RAG) que permitirán enriquecer las respuestas del chatbot al aprovechar la información contenida en documentos existentes sobre TFGs. Información que será debidamente embebida en vectores(en el espacio del modelo) y almacenada en una base de datos vectorizada. Este enfoque combina la potencia de los LLM con la precisión de la recuperación de información RAG, creando así un asistente virtual altamente competente para este caso de uso.

Además, en aras de evaluar y verificar la eficacia de este nuevo chatbot, se llevará a cabo una comparación con un chatbot preexistente desarrollado en un proyecto de fin de grado anterior. Este análisis nos permitirá identificar las mejoras y ventajas de nuestra nueva aproximación.

En resumen, este proyecto aborda la problemática de acceder a una gran cantidad de información que no es de dominio general a través de la combinación de LLM y técnicas RAG. Se espera no solo crear un chatbot que pueda ser utilizado para este caso de uso sino también generar un proceso que pueda ser aplicado a otros casos de uso similares en los que exista una gran cantidad de información especifica y que no sea de dominio público. 





\capitulo{2}{Objetivos del proyecto}

El objetivo fundamental de este proyecto radica en el desarrollo de un asistente virtual, específicamente un chatbot basado en ``\textit{Large Language Models}'' (LLM), como ChatGPT, Bard o LLama. Este chatbot se concibe como una herramienta avanzada que tiene como misión principal proporcionar orientación, respuestas a preguntas comunes y asistencia en las diferentes fases de la realización de un TFG en Ingeniería Informática. Más allá de la simple automatización de respuestas, este chatbot se diseña con la ambición de convertirse en un recurso confiable y eficaz que brinde apoyo en tiempo real a los estudiantes, facilitándoles la toma de decisiones informadas y la superación de obstáculos.

Sin embargo, la singularidad de este proyecto reside en su enfoque hacia la mejora continua de la precisión y relevancia de las respuestas proporcionadas por el chatbot. Para lograr esto, se implementarán técnicas de ``\textit{Retrieval-augmented Generation}'' (RAG). Estas técnicas permitirán que el chatbot enriquezca sus respuestas mediante la recuperación de información específica sobre TFG contenida en documentos previamente existentes. Estos documentos son PDFs con la regulación de la UBU, Repositorios de TFG previos e información en la pagina web de la unversidad entre otros. Esta combinación de LLM y RAG garantizará que las respuestas sean no solo coherentes y útiles, sino también contextualmente relevantes.

Además de diseñar y desarrollar este chatbot, este proyecto se propone un objetivo de comparación. Se buscará evaluar y contrastar el rendimiento y la eficacia de este nuevo chatbot con un chatbot preexistente que ha sido desarrollado en un proyecto de fin de grado anterior. A través de este análisis comparativo, se pretende identificar y documentar las mejoras y ventajas que esta nueva aproximación aporta al proceso de realización de TFG en Ingeniería Informática.

En resumen, los objetivos de este proyecto son tres:
\begin{enumerate}
    \item Diseño y Desarrollo del Chatbot basado en LLM: El primer objetivo es diseñar y desarrollar un chatbot que utilice Large Language Models para responder a las preguntas y dudas de los estudiantes en relación con la realización de sus Trabajos de Fin de Grado en Ingeniería Informática.
    
    \item Integración de Técnicas de RAG para Mejora de la Precisión: El segundo objetivo es integrar técnicas de \textit{Retrieval-augmented Generation} en el chatbot. Estas técnicas se utilizarán para mejorar la precisión y relevancia de las respuestas del chatbot, aprovechando la información contenida en documentos existentes relacionados con los TFG.

    \item Comparación de Rendimiento con Chatbot Preexistente: El tercer objetivo implica comparar el rendimiento y la eficacia del chatbot desarrollado en este proyecto con un chatbot preexistente que fue creado en un proyecto de fin de grado anterior. Esto permitirá identificar las mejoras y ventajas de la nueva aproximación.
    
\end{enumerate}

En un enfoque más general, mediante la creación de un chatbot que aprovecha las capacidades de los LLM y la implementación de técnicas RAG, se pretende explorar y ampliar las posibles aplicaciones de esta tecnología en diversos sectores. La capacidad de proporcionar respuestas altamente precisas y contextualmente relevantes, respaldadas por datos específicos, presenta un gran potencial que se extiende más allá de los límites de este proyecto académico. Estas aplicaciones pueden ser especialmente beneficiosas en sectores como empresas privadas, administración pública y otros contextos donde la información precisa y la interacción automatizada son fundamentales.






\capitulo{3}{Conceptos teóricos}

Este apartado de conceptos teóricos desempeña un papel importante al condensar los fundamentos esenciales de principios, teorías y términos subyacentes en el dominio de conocimiento relacionado con el proyecto. En este contexto, su propósito principal es brindar una visión panorámica de los conceptos teóricos cruciales que servirán como cimiento para la comprensión y avance de este \acrlong{tfg}. A través de una exposición minuciosa de estos conceptos, se proporciona una base conceptual que permite una comprensión más profunda de su aplicación práctica en el proyecto. Asimismo, se busca definir una serie de términos y establecer una base de conocimiento compartida para las secciones subsiguientes.

En este contexto, este apartado tiene como objetivo ofrecer una visión general de los conceptos teóricos esenciales, con un énfasis particular en los \acrfull{llm} y las técnicas de \acrfull{rag}. Estos conceptos, son muy actuales y están en constante evolución en el ámbito del procesamiento de lenguaje natural y la inteligencia artificial, desempeñan un papel fundamental en la comprensión y avance del proyecto en consideración.

Los \acrshort{llm} representan un hito significativo en el campo de la generación de texto y la comprensión del lenguaje. Modelos como GPT-3 han demostrado una capacidad sin precedentes para comprender y generar texto de manera coherente y relevante, convirtiéndose en herramientas poderosas con diversas aplicaciones.

Las técnicas de \acrshort{rag} amplían aún más el potencial de los \acrshort{llm} al permitir el acceso a información específica en documentos o bases de conocimiento existentes. Esta capacidad de recuperación y generación mejorada se traduce en respuestas más precisas y contextualmente relevantes, lo que resulta particularmente valioso en situaciones que requieren asistencia y generación de contenido.

En el núcleo de los \acrshort{llm} y las técnicas de \acrshort{rag}, se encuentran otros conceptos fundamentales de embeddings y bases de datos vectoriales. Estos conceptos son la columna vertebral que impulsa la capacidad de estos modelos para entender y generar texto de manera efectiva. Los embeddings, representaciones vectoriales de palabras y frases, permiten a los \acrshort{llm} comprender y procesar el lenguaje natural, capturando la semántica y relaciones entre palabras. Por otro lado, las bases de datos vectoriales almacenan información en un espacio vectorial, lo que habilita la recuperación eficiente de información relevante. La interacción sinérgica de estos conceptos permite a los \acrshort{llm} y las técnicas \acrshort{rag} acceder a conocimiento específico y generar respuestas contextualmente enriquecidas, mejorando la precisión y relevancia en la comunicación y generación de contenido.

\section{¿Que es un ``\textit{Large Language Models}''?}

En esencia, un modelo lingüístico de gran escala es un tipo de modelo de aprendizaje automático que puede comprender y generar lenguaje humano mediante redes neuronales profundas (\acrlong{dnn}). La principal tarea de un modelo lingüístico es calcular la probabilidad de que una palabra siga a una entrada dada en una frase: por ejemplo, ``El cielo es ....'', siendo la respuesta más probable ``azul''. El modelo es capaz de predecir la siguiente palabra de una frase tras recibir un amplio conjunto de datos de texto (o corpus). Básicamente, aprende a reconocer distintos patrones en las palabras. De este proceso se obtiene un modelo lingüístico preentrenado.

Si se ajustan un poco, estos modelos pueden tener diversos usos prácticos, como la traducción o la adquisición de conocimientos especializados en un campo concreto, como el Derecho o la Medicina. Este proceso se conoce como aprendizaje por transferencia, que permite a un modelo aplicar los conocimientos adquiridos de una tarea a otra.

Lo que hace que un modelo lingüístico sea ``grande'' es el tamaño de su arquitectura. Ésta, a su vez, se basa en la inteligencia artificial de las redes neuronales, muy parecidas al cerebro humano, donde las neuronas trabajan juntas para aprender de la información y procesarla. Además, los \acrshort{llm} constan de un gran número de parámetros (por ejemplo, \acrshort{gpt} tiene más de 100.000 millones) entrenados en grandes cantidades de datos de texto sin etiquetar mediante aprendizaje autosupervisado o semisupervisado. Con el primero, los modelos son capaces de aprender a partir de texto no anotado, lo que supone una gran ventaja si se tienen en cuenta los costosos inconvenientes de tener que depender de datos etiquetados manualmente.

Además, las redes más grandes y con más parámetros han demostrado un mejor rendimiento, con una mayor capacidad para retener información y reconocer patrones en comparación con sus homólogas más pequeñas. Cuanto mayor es el modelo, más información puede aprender durante el proceso de entrenamiento, lo que a su vez hace que sus predicciones sean más precisas. Aunque esto puede ser cierto en el sentido convencional, hay una salvedad: tanto las empresas de \acrshort{ia} como los desarrolladores están encontrando formas de sortear los retos que plantean los excesivos costes computacionales y la energía necesaria para entrenar los \acrshort{llm} introduciendo modelos más pequeños y entrenados de forma más óptima.

Aunque los \acrshort{llm}  se han entrenado principalmente para tareas sencillas, como predecir la siguiente palabra de una frase, es asombroso ver la cantidad de estructura y significado del lenguaje que han sido capaces de captar, por no mencionar el enorme número de datos que pueden recoger.

\subsection{Historia y desarrollo de los LLM}

La historia y evolución de los \acrshort{llm} se remontan a varias décadas de investigación y desarrollo en el campo del procesamiento de lenguaje natural y la \acrlong{ia}. A continuación, se proporciona un resumen de los hitos más significativos en la historia de los \acrshort{llm}\cite{zhao2023survey,scribbleData,Tolaka}.

\imagen{LLL_Evolution}{Cronología de la evolución de los LLM}{1}

\begin{description}

\item[Década de 1950-1960]: Los primeros pasos en la creación de modelos de lenguaje se remonta a los experimentos pioneros con redes neuronales y sistemas de procesamiento de información neuronal realizados en la década de 1950 con el propósito de permitir a las computadoras comprender y procesar el lenguaje natural. Colaboraciones entre investigadores de IBM y la Universidad de Georgetown dieron lugar a la creación de un sistema capaz de traducir automáticamente frases del ruso al inglés, lo que marcó un hito notable en la traducción automática. A partir de ese punto, la investigación en este campo experimentó un auge significativo. Durante esta misma época, se dieron los primeros pasos en el desarrollo de modelos de lenguaje, con investigadores dedicados a la creación de reglas gramaticales y algoritmos para analizar y generar texto. Sin embargo, estos enfoques iniciales se basaban en reglas manuales y presentaban limitaciones significativas en cuanto a su efectividad.

La idea de los \acrshort{llm} surgió con la creación de \textit{Eliza} en los años 60, fue el primer chatbot del mundo, diseñado por el investigador del \acrshort{mit} Joseph Weizenbaum. \textit{Eliza} marcó el inicio de la investigación sobre el \acrfull{pln} y sentó las bases para futuros \acrshort{llm} más complejos.

\item[Década de 1980-1990]: Se produjeron avances en el \acrfull{pln} con la introducción de modelos estadísticos y técnicas de aprendizaje automático. Modelos como el modelo de lenguaje de Markov oculto (HMM) se convirtieron en populares para tareas de \acrshort{pln}. Una de las innovaciones mas significativas fue la introducción de las redes de memoria a largo plazo (LSTM) en 1997, que permitieron crear redes neuronales más profundas y complejas, capaces de manejar cantidades de datos más significativas.

\item[Década de 2000-2010]: Otro momento crucial fue la suite CoreNLP de Stanford, introducida en 2010. Esta suite ofrecía un conjunto de herramientas y algoritmos que ayudaban a los investigadores a abordar tareas de \acrshort{pln} complejas, como el análisis de sentimientos y el reconocimiento de entidades con nombre. Surgieron también modelos estadísticos más avanzados, como los modelos de lenguaje basados en \acrfull{svm} y las \acrfull{crf}. Estos modelos mejoraron la capacidad de procesar y generar texto de manera más efectiva.

\item[Década de 2010-2020]: Esta década marcó un hito significativo con la llegada de modelos basados en redes neuronales, especialmente modelos de lenguaje recurrente (RNN) y modelos de lenguaje basados en \textit{Transformers}. En 2011, Google Brain hizo su debut, proporcionando a los investigadores un acceso invaluable a recursos informáticos de gran potencia y conjuntos de datos enriquecidos, además de ofrecer características avanzadas, como la incrustación de palabras. Esta innovación permitió a los sistemas de \acrlong{pln} comprender el contexto de las palabras de manera más efectiva. 

El trabajo pionero de Google Brain sentó las bases para avances significativos en el campo, incluyendo la aparición de los modelos \textit{Transformer} en 2017. La arquitectura de los \textit{Transformers} revolucionó la creación de \acrfull{llm} más grandes y sofisticados, ejemplificados por el \acrfull{gpt} de OpenAI. Uno de los modelos más influyentes es el GPT-1 desarrollado por OpenAI en 2018.

GPT-2, una versión más grande y avanzada de GPT-1, causó revuelo en la comunidad de inteligencia artificial debido a su capacidad para generar texto coherente y de alta calidad. OpenAI inicialmente decidió no publicar GPT-2 debido a preocupaciones sobre el uso malicioso. 

Fue en 2019 cuando los investigadores de Google presentaron BERT, el modelo bidireccional de 340 millones de parámetros (el tercer modelo más grande de su clase) que podía determinar el contexto permitiéndole adaptarse a diversas tareas. Al preentrenar a BERT en una amplia variedad de datos no estructurados mediante aprendizaje autosupervisado, el modelo pudo comprender las relaciones entre las palabras. En poco tiempo, BERT se convirtió en la herramienta de referencia para las tareas de procesamiento del lenguaje natural. De hecho, BERT estaba detrás de todas las consultas en inglés realizadas a través de Google Search.

\item[2020 en adelante]: GPT-3, lanzado por OpenAI, marcó un avance significativo en el campo de los \acrshort{llm}. Con 175 mil millones de parámetros, GPT-3 demostró una sorprendente capacidad para comprender y generar texto de manera coherente. Se convirtió en un modelo base para muchas aplicaciones de procesamiento de lenguaje natural y impulso el movimiento basado ``\textit{Generative AI}'' al gran público, especialmente a través de la \textit{interface} Chat-GPT.

Desde GPT-3, la investigación en \acrshort{llm} ha continuado avanzando. Se han desarrollado modelos aún más grandes y efectivos, y se han aplicado a una amplia gama de aplicaciones, desde asistentes virtuales hasta traducción automática y generación creativa de texto o imagen.

La versión más reciente hasta la fecha es GPT-4, que presenta mejoras significativas, como la capacidad de utilizar visión por ordenador para interpretar datos visuales (a diferencia de ChatGPT, que utiliza GPT-3.5). GPT-4 acepta como entrada tanto texto, como imágenes. 

Y lo que es más, el último avance es la \textit{steerability} (direccionabilidad), que permite a los usuarios de \acrshort{gpt} personalizar la estructura de su salida para satisfacer sus necesidades específicas. Básicamente, la direccionabilidad alude a la capacidad de controlar o modificar el comportamiento de un modelo lingüístico, lo que implica hacer que el \acrshort{llm} adopte distintos roles, siga instrucciones del usuario o hable con un tono determinado. La direccionabilidad permite al usuario cambiar el comportamiento de un \acrshort{llm} a voluntad y ordenarle que escriba con un estilo o una voz diferentes. Las posibilidades son infinitas.

\end{description}

\imagen{Timeline_LLM2}{Cronología de los modelos lingüísticos de gran tamaño (superior a 10B) de los últimos años}{1}

Es importante destacar que la evolución de los \acrshort{llm} ha sido impulsada en gran medida por el aumento en la disponibilidad de datos de entrenamiento, el desarrollo de arquitecturas de redes neuronales más avanzadas y la mejora en el hardware de cómputo. Estos avances han permitido a los \acrshort{llm} alcanzar un nivel de comprensión y generación de texto que antes era impensable.

\section{Como funciona un LLM}



\subsection{Word2Vec}

Word2Vec fue un algoritmo revolucionario en el \acrfull{pln} que se emplea para aprender representaciones vectoriales densas de palabras a partir de grandes cantidades de texto. Desarrollado por un equipo de investigadores de Google en 2013\cite{Mikolov2013Word2Vec}, este enfoque no supervisado ha tenido un impacto significativo en el campo del \acrshort{pln}. Construyeron un modelo para incrustar palabras en un espacio vectorial, un problema que ya contaba con una larga historia académica en aquel momento, que comenzaba en la década de 1980. Su modelo utilizaba un objetivo de optimización diseñado para convertir las relaciones de correlación entre palabras en relaciones de distancia en el espacio de incrustación: se asociaba un vector a cada palabra de un vocabulario, y los vectores se optimizaban para que el producto punto (proximidad del coseno) entre vectores que representaban palabras que coincidían con frecuencia estuviera más cerca de 1, mientras que el producto punto entre vectores que representaban palabras que rara vez coincidían estuviera más cerca de 0. Descubrieron que el espacio de incrustación resultante era un espacio vectorial. 

El espacio de incrustación resultante hacía mucho más que captar la similitud semántica. Presentaba alguna forma de aprendizaje emergente: era capaz de realizar ``aritmética de palabras'', algo para lo que no había sido entrenado. Existía un vector en el espacio que podía añadirse a cualquier sustantivo masculino para obtener un punto cercano a su equivalente femenino. Por ejemplo, V(rey) - V(hombre) + V(mujer) = V(reina). Un ``vector de género". Esta capacidad de realizar operaciones matemáticas en el espacio de incrustación reveló una comprensión subyacente de las relaciones semánticas. Parecía haber docenas de vectores mágicos de este tipo: un vector plural, un vector para pasar de nombres de animales salvajes a su equivalente más cercano en mascotas, y muchos otros. Estos descubrimientos abrieron nuevas perspectivas en el campo del procesamiento de lenguaje natural y subrayaron la potencia de Word2Vec en la representación de palabras\cite{Chollet}.

\imagen{Word2Vec}{Ilustración de un espacio de incrustación 2D tal que el vector que une ``lobo'' con ``perro'' es el mismo que el vector que une ``tigre'' con ``gato''.}{0.5}

Word2Vec se implementa en dos arquitecturas principales: \acrfull{cbow} y \textit{Skip-gram}. El modelo \acrshort{cbow} se utiliza para predecir una palabra objetivo basada en un contexto circundante, mientras que el modelo \textit{Skip-gram} realiza predicciones inversas, es decir, predice palabras de contexto a partir de una palabra dada. Estas arquitecturas han demostrado ser altamente efectivas en la generación de representaciones vectoriales de palabras que capturan significados y relaciones con precisión.

Las representaciones de palabras aprendidas con Word2Vec se han convertido en una pieza fundamental en tareas de transferencia de conocimiento en \acrshoert
pln, lo que ha impulsado su amplia adopción en la comunidad de investigación y desarrollo. A través de su capacidad para reducir la dimensionalidad y su habilidad para mejorar el rendimiento en tareas de procesamiento de lenguaje, Word2Vec ha dejado una huella perdurable en la forma en que las computadoras comprenden y utilizan el lenguaje natural en una variedad de aplicaciones.

\section{Tipos de LLM}

\imagen{TiposLLM}{Clasificación de los LLM en tres categorias}{1}

\section{¿Que es un \textit{Retrieval-augmented Generation}?}

Los LLM han demostrado su capacidad para comprender el contexto y ofrecer respuestas precisas a diversas tareas de procesamiento de lenguaje natural (PLN), como la síntesis o las preguntas y respuestas, cuando se les solicita. Aunque son capaces de ofrecer muy buenas respuestas a preguntas sobre información con la que fueron entrenados, tienden a ``alucinar'' cuando el tema trata sobre información que desconocen, es decir, que no estaba incluida en sus datos de entrenamiento.

``\textit{Retrieval-augmented Generation}'' (RAG) es una técnica avanzada en el campo del PNL que combina dos enfoques clave: recuperación y generación de texto. Esta técnica se utiliza para mejorar la generación de texto automática y garantizar que las respuestas generadas sean precisas, relevantes y contextualmente adecuadas\cite{Lewis2020}.

En un sistema de RAG, el proceso se divide en dos etapas:

\begin{enumerate}
    \item Recuperación (\textit{Retrieval}): En esta etapa, el sistema busca información relevante en grandes conjuntos de datos o bases de conocimiento. Utiliza métodos de recuperación de información para encontrar documentos o fragmentos de texto que contienen información relacionada con la consulta o el contexto actual.

    \item Generación (\textit{Generation}): Una vez que se ha recuperado la información relevante, el sistema de generación de texto (a menudo basado en un LLM, como GPT) utiliza esta información para generar respuestas coherentes y contextualmente apropiadas.
    
\end{enumerate}

La combinación de estas dos etapas permite que la técnica RAG proporcione respuestas que no solo se basen en el conocimiento preexistente\cite{chen-etal-2017-reading}, sino que también sean sensibles al contexto específico de la consulta o la tarea. Esto propicia respuestas más precisas y relevantes en comparación con enfoques puramente generativos\cite{fan-etal-2019-eli5,hossain-etal-2020-simple}.

RAG se utiliza en una variedad de aplicaciones, incluyendo chatbots, sistemas de respuesta automática, motores de búsqueda mejorados y generación de contenido automático, donde la capacidad de acceder y utilizar información específica es esencial para brindar respuestas mas precisas.

\section{Embeddings}

\section{Bases de datos Vectoriales}

\section{Selección de un LLM}

Aspectos a considerar: Precio, tamaño, funciones.

\subsection{OpenAI}

\imagen{OpenAI}{Evolución tecnológica de los modelos GPT}{1}

\subsection{LLaMa}

\cite{Murtuza}

\imagen{LLaMa}{Gráfico de la evolución de los modelos LLaMa}{1}

\subsection{Bard}

\subsection{Mistral}

\cite{Zhong2023AGIEvalAH}
\imagen{Mistral_Comparacion}{Resultados en MMLU, Razonamiento de sentido común, Conocimiento del mundo y Comprensión lectora para Mistral 7B y Llama 2 (7B/13/70B).}{1}

\subsection{Otros LLM}

MiniLLMs

\section{Prompt Engineering}

\section{Tablas}

Igualmente se pueden usar los comandos específicos de \LaTeX o bien usar alguno de los comandos de la plantilla.

\tablaSmall{Herramientas y tecnologías utilizadas en cada parte del proyecto}{l c c c c}{herramientasportipodeuso}
{ \multicolumn{1}{l}{Herramientas} & App AngularJS & API REST & BD & Memoria \\}{ 
HTML5 & X & & &\\
CSS3 & X & & &\\
BOOTSTRAP & X & & &\\
JavaScript & X & & &\\
AngularJS & X & & &\\
Bower & X & & &\\
PHP & & X & &\\
Karma + Jasmine & X & & &\\
Slim framework & & X & &\\
Idiorm & & X & &\\
Composer & & X & &\\
JSON & X & X & &\\
PhpStorm & X & X & &\\
MySQL & & & X &\\
PhpMyAdmin & & & X &\\
Git + BitBucket & X & X & X & X\\
Mik\TeX{} & & & & X\\
\TeX{}Maker & & & & X\\
Astah & & & & X\\
Balsamiq Mockups & X & & &\\
VersionOne & X & X & X & X\\
} 

\capitulo{4}{Técnicas y herramientas}

En esta sección de la memoria, se presentan las técnicas y las herramientas de desarrollo que han sido empleadas en la ejecución de este proyecto. Debido a que el campo de los \acrshort{llm} se encuentra actualmente en una fase de desarrollo temprana y en constante cambio, se han barajado distintas alternativas para la realización de este proyecto. El carácter investigador que se ha seguido, hace que se hayan seguido distintos caminos que se han tenido que desechar al ir encontrando limitaciones o mejores soluciones disponibles.

Durante la etapa de prototipado y la fase inicial de investigación, se emplearon herramientas distintas a las utilizadas en la versión definitiva del proyecto. Las herramientas utilizadas más destacadas se describen en los siguientes apartados.

\section{Específicas de los LLM}

A pesar de que la inteligencia artificial generativa es un campo relativamente reciente, la explosión de los \acrshort{llm} ha dado lugar a uno numero considerable de herramientas y técnicas en constante cambio y desarrollo actualmente.

Para la gestión del \acrshort{llm} se han barajado en distintos momentos alternativas y ampliaciones a Langchain como el uso LlamaIndex o GPT4all. Se ha optada finalmente por combinar Langchain con Hugging Face, ambas herramientas se explican mas adelante en detalla, aunque también se explican algunas herramientas para el uso de \acrshort{llm} intalados localmente.

\subsection{LangChain}

LangChain es un \textit{framework} de código abierto que permite a los desarrolladores de software que trabajan con \acrfull{ia} y su subconjunto de aprendizaje automático combinar \acrlong{llm} con otros componentes externos para desarrollar aplicaciones impulsadas por modelos \acrshort{llm}. El objetivo de LangChain es vincular modelos de \acrshort{llm}, como GPT-3.5 y GPT-4 de OpenAI, con una variedad de fuentes de datos externas para crear y aprovechar los beneficios de las aplicaciones de \acrfull{pln}.

Desarrolladores, ingenieros de software y científicos de datos con experiencia en los lenguajes de programación Python, JavaScript o TypeScript pueden utilizar los paquetes de LangChain ofrecidos en esos idiomas. LangChain se lanzó como un proyecto de código abierto por los cofundadores Harrison Chase y Ankush Gola en 2022; la versión inicial se lanzó ese mismo año\cite{Langchain}.

\subsubsection{¿Por qué es importante LangChain?}
LangChain es un \textit{framework} que simplifica el proceso de creación de interfaces de aplicaciones de inteligencia artificial generativa. Los desarrolladores que trabajan en este tipo de interfaces utilizan diversas herramientas para crear aplicaciones avanzadas de \acrshort{pln}; LangChain agiliza este proceso. Por ejemplo, los \acrshort{llm} deben acceder a grandes volúmenes de big data, por lo que LangChain organiza estas grandes cantidades de datos para que puedan accederse fácilmente.

Además, los modelos \acrfull{gpt} generalmente se entrenan en datos hasta su liberación al público. Por ejemplo, ChatGPT se lanzó al público a finales de 2022, pero su base de conocimientos se limitaba a datos de 2021 y anteriores. LangChain puede conectar modelos de \acrshort{ia} a fuentes de datos para darles conocimiento de datos recientes sin limitaciones.

\subsubsection{¿Cuáles son las características de LangChain?}
LangChain se compone de los siguientes módulos que aseguran que los múltiples componentes necesarios para crear una aplicación efectiva de \acrshort{pln} puedan funcionar sin problemas:
\begin{enumerate}

\item \textbf{Interacción del modelo:} También llamado entrada/salida del modelo, este módulo permite que LangChain interactúe con cualquier modelo de lenguaje y realice tareas como gestionar las entradas al modelo y extraer información de sus salidas.
Conexión y recuperación de datos. Los datos a los que acceden los \acrshort{llm} pueden transformarse, almacenarse en bases de datos y recuperarse de esas bases de datos mediante consultas con este módulo.
\item \textbf{Cadenas:} Al construir aplicaciones más complejas con LangChain, se pueden requerir otros componentes o incluso más de un \acrshort{llm}. Este módulo enlaza múltiples \acrshort{llm} con otros componentes o \acrshort{llm}, conocido como una cadena de \acrshort{llm}.
\item \textbf{Agentes:} El módulo de agentes permite que los \acrshort{llm} decidan los mejores pasos o acciones a tomar para resolver problemas. Lo hace orquestando una serie de comandos complejos a los \acrshort{llm} y otras herramientas para que respondan a solicitudes específicas.
\item \textbf{Memoria:} El módulo de memoria ayuda a un \acrshort{llm} a recordar el contexto de sus interacciones con los usuarios. Se puede agregar memoria a corto y largo plazo a un modelo, según el uso específico.
\end{enumerate}

\imagen{langchain}{Esquema del \textit{Framework} de Langchain con los distintos componentes y modulos.}{1}

\subsubsection{¿Cuáles son las integraciones de LangChain?}
LangChain generalmente construye aplicaciones utilizando integraciones con proveedores de \acrshort{llm} y fuentes externas donde se pueden encontrar y almacenar datos. Por ejemplo, LangChain puede construir chatbots o sistemas de preguntas y respuestas integrando un \acrshort{llm}, como los de Hugging Face, Cohere y OpenAI, con fuentes o almacenes de datos como Apify Actors, Google Search y Wikipedia. Esto permite que una aplicación tome texto de entrada del usuario, lo procese y recupere las mejores respuestas de cualquiera de estas fuentes. En este sentido, las integraciones de LangChain utilizan la tecnología de \acrlong{pln} más actualizada para construir aplicaciones efectivas.

Otras integraciones potenciales incluyen plataformas de almacenamiento en la nube, como Amazon Web Services, Google Cloud y Microsoft Azure, así como bases de datos de vectores. Una base de datos de vectores puede almacenar grandes volúmenes de datos de alta dimensión, como videos, imágenes y texto extenso, como representaciones matemáticas que facilitan la consulta y búsqueda de esos elementos de datos. Pinecone es un ejemplo de base de datos de vectores que se puede integrar con LangChain.

\subsubsection{¿Cómo crear prompts en LangChain?}
Los prompts sirven como entrada al \acrshort{llm} que le indica que devuelva una respuesta, que suele ser una respuesta a una consulta. Esta respuesta también se denomina salida. Un prompt debe diseñarse y ejecutarse correctamente para aumentar la probabilidad de obtener una respuesta bien escrita y precisa de un modelo de lenguaje. Es por eso que la ingeniería de prompts es una ciencia emergente que ha recibido más atención en los últimos años.

Los prompts pueden generarse fácilmente en implementaciones de LangChain utilizando una plantilla de prompt, que se utilizará como instrucciones para el \acrshort{llm} subyacente. Las plantillas de prompts pueden variar en especificidad. Pueden diseñarse para plantear preguntas simples a un modelo de lenguaje. También se pueden utilizar para proporcionar un conjunto de instrucciones explícitas a un modelo de lenguaje con suficiente detalle y ejemplos para recuperar una respuesta de alta calidad.

LangChain generalmente requiere al menos una integración. OpenAI es un ejemplo destacado. Para usar las interfaces de programación de aplicaciones \acrshort{llm} de OpenAI, un desarrollador debe crear una cuenta en el sitio web de OpenAI y recuperar la clave de acceso a la \acrshort{api}. Luego, utilizando el siguiente fragmento de código, instale el paquete Python de OpenAI e ingrese la clave para acceder a las \acrshort{api}.

\subsubsection{¿Cómo desarrollar aplicaciones en LangChain?}
LangChain está diseñado para desarrollar aplicaciones con funcionalidad de modelos de lenguaje. Hay diferentes formas de hacer esto, pero el proceso generalmente implica algunos pasos clave.

El desarrollador debe definir primero un caso de uso específico para la aplicación. Esto también implica determinar su alcance, incluidos los requisitos como cualquier integración, componente y \acrshort{llm} necesario.
Construir funcionalidad. Los desarrolladores utilizan prompts para construir la funcionalidad o lógica de la aplicación prevista.

LangChain permite a los desarrolladores modificar su código para crear funcionalidades personalizadas que satisfagan las necesidades del caso de uso y den forma al comportamiento de la aplicación. Aunque es cierto que solo se puede modificar hasta un cierto punto. Es importante elegir el \acrshort{llm} adecuado para el trabajo y también ajustarlo finamente para cumplir con las necesidades del caso de uso.

\subsection{Hugging Face}

Hugging Face es una empresa y plataforma que se especializa en modelos de lenguaje natural y \acrlong{dnn}. Ofrecen una amplia gama de recursos y herramientas destinados a facilitar el desarrollo, entrenamiento y despliegue de modelos de \acrfull{pln}).

Algunos aspectos destacados de Hugging Face incluyen\cite{HuggingFace}:
\begin{itemize}

\item \textbf{Modelos preentrenados:} Hugging Face proporciona acceso a una variedad de modelos de lenguaje natural preentrenados de última generación. Esto incluye modelos como BERT, \acrshort{gpt}, Mistral, LLaMa, RoBERTa, y muchos otros, que han demostrado un rendimiento excepcional en diversas tareas de procesamiento del lenguaje natural.

\item \textbf{Transformers Library:} La Transformers Library de Hugging Face es una biblioteca de código abierto que facilita el uso, entrenamiento y ajuste fino de modelos de transformer para tareas específicas. Proporciona una interfaz consistente para varios modelos y se utiliza ampliamente en la comunidad de aprendizaje profundo para \acrshort{pln}.

\item \textbf{Hugging Face Hub:} Es una plataforma en línea que permite a los desarrolladores compartir, explorar y utilizar modelos de lenguaje natural de Hugging Face. Facilita la colaboración y el intercambio de modelos entrenados por la comunidad.

\item \textbf{Pipeline API:} Hugging Face ofrece una \acrshort{api} de canalización (Pipeline API) que simplifica el uso de modelos complejos para tareas específicas. Esto hace que sea fácil utilizar modelos de \acrshort{pln} preentrenados para clasificación de texto, traducción, resumen y más.

\item \textbf{Comunidad de usuarios:} La plataforma fomenta la colaboración y la contribución de la comunidad al código y los modelos. Los usuarios pueden contribuir con modelos, compartir implementaciones y participar en discusiones relacionadas con el procesamiento del lenguaje natural y la \acrlong{ia}.
\end{itemize}

En resumen, Hugging Face se ha convertido en un recurso integral para la comunidad de aprendizaje profundo y \acrshort{pln}, proporcionando modelos avanzados, bibliotecas de código abierto y una plataforma para compartir y colaborar en proyectos relacionados con el \acrlong{pln}.



\subsection{\textit{Quantization y modelos locales}}

El tener que depender de \acrshort{api} gratuitas de \textit{Open Source} ha sido una constante limitación en el proyecto. Si bien estos modelos y herramientas ofrecen bastantes posibilidades, también tienen grandes limitaciones sobre todo en comparación a la \acrshort{api} de OpenAI.

Por este motivo se ha investigado y probado la instalación local de \acrshort{llm} de Open Source como LLaMa2 ha traves de GPT4all y llama-cpp-python.

\subsubsection{Quantization}

GGML es una biblioteca Tensor para \textit{machine learning} que se presenta como una biblioteca en C++. Su función principal es permitir la ejecución de modelos de \acrlong{llm} en la \acrshort{cpu} o en combinación con la \acrshort{gpu}. Un aspecto distintivo de GGML es su definición de un formato binario para la distribución de \acrshort{llm}. Además, GGML emplea una técnica llamada \textit{Quantization} que posibilita la ejecución de modelos de \acrshort{llm} en hardware de consumo\cite{GGML_Gimmi}.

Los pesos de los \acrshort{llm} son números de punto flotante (decimales). Al igual que se necesita más espacio para representar un número entero grande (por ejemplo, 1000) en comparación con un entero pequeño (por ejemplo, 1), se requiere más espacio para representar un número de punto flotante de alta precisión (por ejemplo, 0.0001) en comparación con un número de punto flotante de baja precisión (por ejemplo, 0.1). El proceso de \textit{Quantization} de un modelo de lenguaje grande implica reducir la precisión con la que se representan los pesos para disminuir los recursos necesarios para utilizar el modelo. GGML admite varias estrategias de \textit{Quantization} (por ejemplo, \textit{Quantization} de 4 bits, 5 bits y 8 bits), cada una de las cuales ofrece diferentes compromisos entre eficiencia y rendimiento.

Para utilizar eficazmente los modelos, es esencial tener en cuenta los requisitos de memoria y disco. Dado que los modelos se cargan completamente en la memoria, se necesita suficiente espacio en disco para almacenarlos y RAM suficiente para cargarlos durante la ejecución. En el caso del modelo de 65 mil millones de parámetros, incluso después de la \textit{Quantization}, se recomienda tener al menos 40 gigabytes de RAM disponible. Cabe destacar que los requisitos de memoria y disco son actualmente equivalentes.

\imagen{QuantizedSizeLlama}{Efecto de la \textit{Quantization} en el tamaño de los LLM de LLaMa.}{1}

La \textit{Quantization} desempeña un papel crucial en el manejo de estas demandas de recursos. A menos que se disponga de recursos computacionales excepcionales, la \textit{Quantization} permite utilizar los modelos en configuraciones de hardware más modestas al reducir la precisión de los parámetros del modelo y optimizar el uso de la memoria. Esto garantiza que la ejecución de los modelos siga siendo factible y eficiente para una gama más amplia de configuraciones.

\subsubsection{Llama-cpp-python}

LLama.cpp fue desarrollado por Georgi Gerganov. Implementa la arquitectura LLaMa de Meta de manera eficiente en C/C++ y es una de las comunidades de código abierto más dinámicas en torno a la inferencia de \acrlong{llm} con más de 390 contribuyentes, 43,000+ estrellas en el repositorio oficial de GitHub y 930+ versiones\cite{llama.cpp}.

El diseño de Llama.cpp como una biblioteca C++ centrada en la \acrshort{cpu} significa menos complejidad y una integración perfecta en otros entornos de programación. Esta amplia compatibilidad aceleró su adopción en diversas plataformas. Actuando como un repositorio para características críticas de bajo nivel, Llama.cpp refleja el enfoque de LangChain para capacidades de alto nivel, simplificando el proceso de desarrollo aunque con posibles desafíos de escalabilidad futura.

Se centra en una única arquitectura de modelo, permitiendo mejoras precisas y efectivas. Su compromiso con los modelos Llama a través de formatos como GGML y GGUF ha llevado a ganancias de eficiencia sustanciales.

El núcleo de LLama.cpp son los modelos Llama originales, que también se basan en la arquitectura de \textit{transformers}. Los autores de Llama aprovechan varias mejoras que posteriormente se propusieron y utilizan diferentes modelos como PaLM.

\subsubsection{GPT4All}

GPT4All es un ecosistema diseñado para entrenar e implementar \acrlong{llm}. Notablemente, estos modelos están destinados a ejecutarse localmente en \acrshort{cpu} de consumo, lo que los hace accesibles y eficientes para una amplia gama de usuarios. El objetivo principal de GPT4All es servir como un modelo de lenguaje ajustado finamente al estilo de un asistente, ofreciendo un alto nivel de personalización. Se alienta a los usuarios, ya sean individuos o empresas, a utilizar, distribuir y construir libremente sobre los modelos de GPT4All\cite{gpt4all}.

Características Clave:
\begin{itemize}

\item \textbf{Implementación Local:} Los modelos de GPT4All están optimizados para ejecutarse en \acrshort{cpu} de consumo, lo que permite la implementación local sin la necesidad de recursos computacionales extensivos.
  
\item \textbf{Personalización:} El ecosistema enfatiza la personalización, permitiendo a los usuarios adaptar los modelos de lenguaje según sus necesidades y preferencias específicas.

\item \textbf{Ecosistema de Código Abierto:} GPT4All está respaldado por un software de ecosistema de código abierto. Este enfoque abierto fomenta la colaboración, la transparencia y las contribuciones de la comunidad.

\item \textbf{Tamaño del Archivo:} Los modelos de GPT4All se distribuyen como archivos descargables que van desde 3 GB hasta 8 GB, lo que facilita su descarga e integración en los sistemas de los usuarios.

\item \textbf{Nomic AI:} El ecosistema de software es respaldado y mantenido por Nomic AI, que desempeña un papel crucial en garantizar la calidad y seguridad de los modelos. Nomic AI también lidera los esfuerzos para simplificar el proceso de entrenamiento e implementación para usuarios que desean crear sus propios modelos de lenguaje amplio.
\end{itemize}

\subsection{FAISS}

FAISS, que significa \textit{Facebook AI Similarity Search} (Búsqueda de Similitud de Inteligencia Artificial de Facebook)\cite{Faiss}, es una biblioteca desarrollada por Facebook(ahora META) para realizar búsquedas eficientes de similitud en conjuntos grandes de datos. Se utiliza comúnmente para realizar búsquedas de vecinos más cercanos en conjuntos de datos de vectores de alta dimensionalidad, como los que se encuentran en tareas de aprendizaje automático y \acrlong{pln}.

Está optimizado para manejar grandes cantidades de datos y realizar búsquedas de vecinos más cercanos de manera eficiente. Puede trabajar con vectores de diferentes dimensiones y tipos de datos, haciendo que sea versátil para aplicaciones en una variedad de dominios.

También aprovecha técnicas de implementación eficientes y puede aprovechar la capacidad de procesamiento paralelo de hardware, como \acrshort{gpu}, para acelerar las operaciones de búsqueda de similitud. Ofrece métodos eficientes para la indexación de vectores, lo que facilita la búsqueda rápida en grandes conjuntos de datos.

Se utiliza a menudo en conjunto con bibliotecas de aprendizaje profundo, como PyTorch, y puede integrarse fácilmente en flujos de trabajo de aprendizaje automático. También se encuentra disponible en bibliotecas de \acrshort{llm} como por ejemplo Langchain. 

Implementa algoritmos especializados para la búsqueda eficiente de vecinos más cercanos en espacios de alta dimensión, lo que la hace destacar en la creación de bases de datos vectoriales, especialmente si se usan las técnicas \acrshort{rag}.

En resumen, FAISS es una herramienta fundamental para tareas que involucran la búsqueda eficiente de similitud en grandes conjuntos de datos, lo que lo convierte en una opción popular en el campo de la recuperación de información, bases de datos vectoriales y la minería de datos.


\section{Prototipado, documentación y gestión}

\subsection{Kaggle}

Para el prototipado se ha hecho uso de Kaggle. Esta elección se ha debido a que los modelos de Mistral y Meta estaban disponibles en la plataforma y existe una amplia comunidad de usuarios.

Kaggle es una plataforma en línea que aloja competiciones de ciencia de datos. Fundada en 2010, Kaggle proporciona un entorno donde científicos de datos y profesionales del aprendizaje automático pueden encontrar conjuntos de datos, participar en competiciones, colaborar en proyectos y mejorar sus habilidades en análisis de datos y modelado predictivo.

Algunas características clave de Kaggle incluyen:
\begin{itemize}
\item \textbf{Competiciones de Ciencia de Datos:} Kaggle organiza competiciones regulares en las que los participantes compiten para resolver problemas de ciencia de datos planteados por empresas o instituciones. Estos desafíos abarcan una amplia gama de temas, desde reconocimiento de imágenes hasta predicción de precios.

\item \textbf{Conjuntos de Datos Públicos:} Kaggle proporciona un repositorio de conjuntos de datos públicos que los usuarios pueden explorar y utilizar para prácticas y proyectos. Estos conjuntos de datos abarcan diversas áreas, desde datos económicos hasta imágenes médicas.

\item \textbf{Kernels:} Los Kernels son entornos de desarrollo en línea que permiten a los usuarios escribir, ejecutar y compartir código en lenguajes como Python y R. Los Kernels son útiles para la exploración de datos y la creación de modelos.

\item \textbf{Foros y Comunidad:} Kaggle cuenta con una comunidad activa de científicos de datos y profesionales del aprendizaje automático. Los foros permiten la discusión de problemas, la obtención de asesoramiento y el intercambio de conocimientos.

\item \textbf{Aprendizaje y Recursos:} Kaggle ofrece tutoriales, cursos y recursos educativos para ayudar a los usuarios a mejorar sus habilidades en ciencia de datos y aprendizaje automático.
\end{itemize}

En general, Kaggle ha crecido hasta convertirse en una de las plataformas más importantes para la comunidad de ciencia de datos y el uso d modelos, proporcionando oportunidades para la colaboración, la competencia y el aprendizaje continuo.

\subsection{Github}

Github ha sido la herramienta de repositorio de versiones para el proyecto. Aunque se han valorado otras opciones como Gitlab, se ha optado por Github al ser una herramienta conocida de amplio uso en el grado de ingeniería informática en la \acrshort{ubu}.

GitHub es una plataforma de desarrollo de software basada en web que utiliza el sistema de control de versiones Git. Lanzada en 2008, se ha convertido en una de las plataformas más populares para el alojamiento de proyectos de desarrollo colaborativo. Algunos aspectos clave de GitHub incluyen:

Permite a los desarrolladores alojar sus proyectos y controlar las versiones de su código utilizando Git. Los repositorios pueden ser públicos (accesibles para todos) o privados (restringidos a un conjunto de colaboradores). Facilita la colaboración entre desarrolladores. Varios colaboradores pueden trabajar en un proyecto, realizar cambios y fusionar sus contribuciones de manera eficiente.

Utiliza Git para el control de versiones, lo que permite realizar un seguimiento de los cambios en el código a lo largo del tiempo. Los desarrolladores pueden revertir a versiones anteriores, ramificar su código para trabajar en nuevas características y fusionar cambios de diferentes ramas.

GitHub proporciona herramientas para realizar un seguimiento de problemas (bugs, mejoras, tareas, etc.) y para proponer cambios en el código mediante solicitudes de extracción. Puede integrarse con servicios de despliegue continuo, lo que facilita la implementación automática de cambios en un entorno de producción después de pasar las pruebas necesarias.

\subsection{Zube.io}

Zube es una potente herramienta de gestión de proyectos y colaboración que ha facilitado mi gestión del proyecto. Al principio se probaron otras posibles alternativas mas integradas en Github, como Zenhub o Proyects, pero al final se opto por Zube.io debido a los siguientes motivos:
\begin{enumerate}

\item \textbf{Tablero Kanban vs. Tablero Sprint:} Zube admite Kanban y Sprint, lo que permite a los usuarios elegir uno de ellos para la gestión de proyectos, y ambos métodos pueden utilizarse al mismo tiempo. La principal diferencia entre Kanban y Sprint es que Sprint tiene una línea de tiempo predefinida, generalmente de 2 semanas. Durante este periodo, se lleva a cabo una discusión diaria para alinear el estado del proyecto y abordar cualquier caso urgente. Cada 2 semanas se completan algunas características para que se prueben o revisen antes de pasar al siguiente sprint. Kanban no tiene el concepto de línea de tiempo, pero es útil para proyectos a largo plazo.

\imagen{SprintBoard}{\textit{Sprint Board} de Zube.io con los distintos estados de las \textit{Issues}.}{1}

\item \textbf{Conexión con Github:} Una de las mejores características de Zube es que puede conectarse con Github, una popular plataforma de gestión de código. Esta conexión permite que los desarrolladores gestionen su código en Github, y los Project Managers (PM) pueden verificar el estado en Zube, facilitando una estrecha colaboración sin interferir en el trabajo del otro.

\item \textbf{Etiquetas:} Se pueden crear y editar etiquetas tanto en Zube como en Github. Aunque es una función simple, las etiquetas son útiles para verificar el estado de las funciones y las tarjetas. Las etiquetas con colores facilitan su distinción en el montón de tarjetas, brindando una ayuda visual.

\item \textit{\textbf{Issue Manager:}} Es una función que ahorra tiempo al mover varias tarjetas a la vez, especialmente al finalizar un sprint. Permite filtrar según cada componente establecido en las tarjetas, actualizar sus parámetros o moverlas directamente a secciones o sprints específicos.

\end{enumerate}

En resumen, Zube simplifica la gestión de proyectos Agile o Scrum al ofrecer opciones flexibles de Kanban, conexión con Github, etiquetas visuales y un eficiente \textit{Issue Manager}.

\subsection{Overleaf}

Para la realización de la documentación de la memoria, se ha optado por usar Overleaf y la plantilla LaTeX del \acrshort{tfg}.

Overleaf es una plataforma en línea diseñada para simplificar la creación, edición y colaboración en documentos científicos y técnicos, especialmente aquellos redactados en LaTeX. Su entorno de escritura colaborativa permite a varios usuarios editar simultáneamente un mismo documento, facilitando la colaboración en proyectos de investigación, artículos científicos, tesis, entre otros. Al utilizar una interfaz basada en web, Overleaf elimina la necesidad de instalar software adicional, permitiendo la edición directa de documentos LaTeX sin requerir una instalación local.

La plataforma ofrece diversas plantillas predefinidas para diferentes tipos de documentos académicos y científicos, junto con herramientas integradas para el formato automático según las convenciones de estilo de LaTeX. Además, proporciona un entorno de compilación en línea que genera el documento final en formato PDF, sin requerir la instalación de compiladores LaTeX en las máquinas locales de los usuarios.

Overleaf facilita la organización de documentos en proyectos, lo que simplifica la gestión de múltiples archivos y carpetas relacionadas. Además, incorpora un sistema de control de versiones que permite realizar un seguimiento de los cambios realizados en el documento a lo largo del tiempo. En resumen, Overleaf se presenta como una herramienta integral para la redacción colaborativa y la creación eficiente de documentos científicos y técnicos.

Para la realización de textos largos, como el del \acrshort{tfg}, el tiempo de compilación de la versión gratuita no es suficiente. Por ello se ha recurrido a la colaboración con las cuentas de pago, que tiene mas funciones, aparte del tiempo de compilación extendido.

\section{Frontend}

Para la creación de la \acrshort{ui} se ha optado por separar en la medida de lo posible el \textit{backend} y \textit{frontend}. Para ello se ha investigado el uso de FastAPI y de una biblioteca de Python llamada Streamlit.

\subsection{FastAPI}

FastAPI es un \textit{framework} para el desarrollo rápido de \acrshort{api} con Python 3.7 (o superior). Diseñado para ser fácil de usar, rápido y eficiente, FastAPI destaca por su sintaxis declarativa, generación automática de documentación interactiva (compatible con Swagger y ReDoc), y la capacidad de aprovechar al máximo las características de la programación asíncrona. Es una elección muy popular para el desarrollo de RESTful APIs y aplicaciones web.

Una de las principales ventajas de las REST API radica en que el protocolo REST separa el almacenamiento de datos (backend) y la interfaz de usuario (frontend) del servidor, lo que permite que el cliente y el servidor sean independientes entre sí. Esta separación es fundamental para la arquitectura REST y aporta beneficios significativos al desarrollo de aplicaciones.

La independencia entre el backend y el frontend facilita la escalabilidad y la flexibilidad del sistema. Al dividir las responsabilidades de manera clara, los equipos de desarrollo pueden trabajar de manera más eficiente y concurrente en las distintas capas de la aplicación. Por ejemplo, el equipo encargado del desarrollo del backend puede realizar mejoras o modificaciones en la lógica de negocio y en la gestión de datos sin afectar directamente a la interfaz de usuario.

Esta separación también favorece la reutilización de componentes y la portabilidad del software. Dado que el backend y el frontend operan de manera independiente, es posible implementar cambios o actualizaciones en una capa sin afectar a la otra, siempre y cuando se respeten los contratos definidos por la REST API. Esto simplifica el mantenimiento y permite la integración de nuevas funcionalidades sin perturbar el funcionamiento existente.

Las características clave de FastAPI incluyen:

\begin{itemize}

\item Rápido y Eficiente: Aprovecha las características de Python 3.7+ para proporcionar un rendimiento excepcional.

\item Sintaxis Declarativa: Utiliza anotaciones de tipo estándar de Python para definir los tipos de datos de entrada y salida de las funciones, lo que facilita la validación y la generación automática de documentación.

\item Automatización de Documentación: FastAPI genera automáticamente documentación interactiva basada en las anotaciones de tipo, lo que facilita a los desarrolladores comprender y probar rápidamente la \acrshort{api}.

\item Compatibilidad con Estándares Abiertos: Es compatible con estándares abiertos como OpenAPI (usando Swagger UI y ReDoc) y JSON Schema.

\item Programación Asíncrona: Aprovecha las características de la programación asíncrona para manejar de manera eficiente un gran número de conexiones concurrentes.

\item Seguridad Integrada: Ofrece funciones integradas para manejar la autenticación, autorización y seguridad en general.

\item Interoperabilidad: Puede integrarse fácilmente con otros marcos y bibliotecas de Python.

\end{itemize}

En resumen, FastAPI es una opción robusta y moderna para el desarrollo de \acrshort{api} en Python, destacándose por su enfoque rápido, sintaxis clara y generación automática de documentación.

\subsection{Streamlit}

Streamlit es una biblioteca de código abierto en Python que se utiliza para crear aplicaciones web interactivas para \textit{data science} y \textit{machine learning} de manera rápida y sencilla. Su objetivo principal es permitir a los usuarios transformar datos en aplicaciones web interactivas con tan solo unas pocas líneas de código.

Con Streamlit, los especialistas de datos y desarrolladores pueden crear fácilmente interfaces de usuario atractivas para sus modelos, visualizaciones y análisis de datos sin necesidad de conocimientos extensos en desarrollo web. La biblioteca se integra bien con bibliotecas populares de Python como Pandas, Matplotlib y Plotly, lo que facilita la creación de aplicaciones web a partir de código existente\cite{Streamlit}.

Streamlit simplifica el proceso de desarrollo de aplicaciones web al manejar automáticamente la actualización de la interfaz de usuario en respuesta a los cambios en los datos subyacentes. Esto permite a los usuarios centrarse más en el análisis de datos y la creación de visualizaciones, sin tener que preocuparse demasiado por los detalles de implementación de la interfaz web.

En resumen, Streamlit es una herramienta valiosa para aquellos que desean hacer una \textit{interface} para análisis de datos y modelos de aprendizaje automático de manera efectiva a través de aplicaciones web interactivas con una curva de aprendizaje mínima.
\capitulo{5}{Aspectos relevantes del desarrollo del proyecto}

En este apartado se mustran los aspectos mas relevantes de este \acrshort{tfg}. Se hace especial hincapié en los decisiones y apartados que mayor influencia han tenido en el proyecto. Por el carácter de investigación del trabajo y por lo temprano del desarrollo de esta tecnología, parte de este apartado está dedicado a la problemática que surge al emplear técnicas y herramientas que en constante cambio.

\section{Selección de un LLM}

Los \acrfull{llm} evolucionan con rapidez y continuamente aparecen nuevos modelos. Esto supone un reto: ¿Cómo seleccionar el \acrshort{llm} más adecuado para este proyecto? En este apartado se analizan las consideraciones prácticas que han guiado el proceso de toma de decisiones.

\subsection{Licencias y uso comercial}
Una consideración crucial a la hora de elegir un \acrshort{llm} es la concesión de licencias. Posiblemente el mas conocido y evolucionado es el \acrshort{gpt} de OpenAI, pero es un modelo que puede tener un coste considerable. En cada \textit{query} se ha de pagar una pequeña cantidad  por token. Para el objeto de este proyecto es mas adecuado el uso de un \acrshort{llm} de \textit{open-source} o por lo menos con una licencia comunitaria para investigación y educación.

Aunque muchos modelos abiertos tienen restricciones de uso comercial, existen modelos disponibles para aplicaciones comerciales. Por ejemplo, la familia de modelos MPT de MosaicML se publica bajo licencias que permiten su uso comercial. Se puede obtener más información sobre las distintas licencias en la \textit{Open Source Initiative} y a traves de la plataforma HugginFace.

\subsection{Factores prácticos para la velocidad de inferencia y la precisión}
Los factores prácticos desempeñan un papel crucial a la hora de determinar la idoneidad de un \acrshort{llm} para el proyecto. Evaluar la velocidad de inferencia (el tiempo que tarda un \acrshort{llm} en procesar y generar resultados) es esencial, sobre todo cuando se trata de grandes cantidades de datos no estructurados. Una inferencia lenta puede dificultar la extracción de información de nuestro chatbot. Optar por modelos optimizados para una inferencia más rápida o capaces de manejar volúmenes de entrada sustanciales puede resultar ventajoso peor también será un modelo mas pesado. Además, si se requiere una gran precisión en el análisis de sentimientos, la selección de un \acrshort{llm} con precisión y análisis de grano fino resulta crucial, y la velocidad de inferencia pasa a ser una consideración secundaria.

\subsection{El impacto de la longitud del contexto y el tamaño del modelo}
Tener en cuenta la longitud del contexto y el tamaño del modelo es crucial a la hora de evaluar los \acrshort{llm}. Mientras que muchos \acrshort{llm} tienen limitaciones en la longitud de entrada, los modelos abiertos más recientes como Salesforce X-Gen admiten longitudes de contexto más largas, lo que permite entradas más completas y resultados deseados. El tamaño del modelo también influye en los requisitos de infraestructura, ya que los modelos más pequeños (menos de siete mil millones de parámetros) son más fáciles de implementar en hardware básico, lo que agiliza la implementación práctica. Por el contrario, algunos \acrshort{llm} ofrecen flexibilidad en el procesamiento de entradas más cortas, pero compensan con parámetros más grandes, atendiendo a casos de uso en los que la precisión dentro de un contexto restringido es primordial. Los \acrshort{llm} con contextos más largos y modelos de mayor tamaño tienden a ser más potentes, pero también tienen mayores exigencias computacionales.

\subsection{Específicos para una tarea o de uso general}
Cuando se trata de \acrfull{llm}, a menudo está la disyuntiva de elegir entre \acrshort{llm} específicos para una tarea o \acrshort{llm} multitarea de propósito general que utilizan \textit{prompts}. Mientras que estos últimos ofrecen versatilidad, los \acrshort{llm} para tareas específicas suelen ser más prácticos y eficientes para caso concretos de uso. Estos modelos especializados se entrenan y ajustan específicamente para una tarea concreta, lo que mejora el rendimiento y la precisión. 

Otro aspecto relacionado es el idioma de entremetimiento del \acrshort{llm}. Los modelos mas extendidos como LLaMa, OpenAI O Mistral, son modelos entrenados principalmente con grandes volúmenes de datos en ingles. Algunos modelos están específicamente entrenados y diseñados para responder en otros idiomas, como Ǎguila\cite{Ǎguila} para español, o multilenguaje como PolyLM\cite{wei2023polylm}.

\subsection{Pruebas y evaluación}
Las pruebas y evaluaciones exhaustivas son cruciales para determinar la fiabilidad de los \acrshort{llm}. Un método eficaz consiste en crear un conjunto de pruebas con ejemplos etiquetados manualmente. Una anotación fiable garantiza mediciones precisas. La comparación de los resultados de los modelos \acrshort{llm} con las referencias etiquetadas ayuda a calcular las métricas de precisión. 

Se hablará mas acerca de este aspecto en una sección posterior, ya que la solución de validación del chatbot dependerá mas del framework de acceso, Langcgain, que del \acrshort{llm} elegido.

\subsection{La revolución de los modelos abiertos}
Recientemente, la adopción de modelos abiertos ha ido en aumento debido a factores como la preocupación por la privacidad de los datos y la rentabilidad. Los modelos abiertos, formados a partir de datos públicos, resuelven los problemas de privacidad asociados a los modelos cerrados. Además, a menudo ofrecen opciones más asequibles. Puede explorar recursos de \acrshort{llm} abiertos en Huggingface y consultar la tabla de clasificación de \acrshort{llm} de \textit{LlamaIndex}\cite{LlamaIndex}.

\subsection{Consideraciones sobre los costes de implementación}
Al seleccionar un \acrshort{llm}, el coste de implementación es también importante, no solo el coste por uso. El tamaño del modelo, los requisitos informáticos y la configuración de la infraestructura influyen en el coste total. Para la escalabilidad, se pueden considerar técnicas de optimización de modelos como la cuantificación(\textit{quantification}), la aceleración de hardware o los servicios cloud para reducir costes. Lograr un equilibrio entre el rendimiento y la asequibilidad de la plataforma es esencial.

Tras realizar algunas pruebas con modelos locales y cuantificación, se optó por usar mejor la \acrshort{api} de HuggingFace. Aunque se tengan algunos retrasos en el establecimiento de la conexión y en momentos puntuales el tiempo de respuesta sea mas lento de lo deseado, se considera que es la mejor opción. La \acrshort{api} permite usar el Chatbot por terceros de forma mas sencilla, sin tener que instalar modelos locales y depender de las características de la \acrshort{cpu} del equipo local.

\subsection{La necesidad de adaptarse al rápido ritmo del cambio}
Casi cada semana se producen cambios en los \acrshort{llm}. Es un momento apasionante pero igualmente supone un reto para desarrollar versiones estables del chatbot. Se seleccionará un \acrshort{llm} que se adapte al caso de uso, y se usarán interfaces de acceso como LangChain que permitan reemplazar el \acrshort{llm} en caso de ser necesario, sin tener que rehacer el chatbot completamente.

Se hablará mas acerca de este aspecto y como ha influido, no solo en la elección del \acrshort{llm} sino también en el resto de aspectos del proyecto.

\subsection{Comparativa de LLMs}

Se han valorado distintos aspectos, y se han probado en mayor o menor medida varios \acrshort{llm}. Los aspectos que se han valorado en mayor profundidad se han recogido en la  tabla \ref{tabla:comparativallm}\cite{Hostinger}\cite{MindsDB}\cite{LlamaIndex}.

\tablaSmall{Comparativa entre distintos \acrlong{llm}}{l c c c c}{comparativallm}
{ \multicolumn{1}{l}{Modelo} & Multilenguaje & API & Open Source & Privacidad \\}{ 
GPT-4 & X & X & &\\
LLaMA 2 & / & / & X &\\
Bard & / & X & &\\
Claude & / & & & /\\
Mistral & / & / & X & X\\
OpenOrca &  & / & X & /\\
} 

\subsection{Modelo elegido: Mistral}

Como ya se ha mencionado anteriormente, Mistral es una Startup Francesa que en solo unos meses ha conseguido una gran repercusión en el mundo de los \acrshort{llm} y del \acrshort{pln}. En comparación con otros modelos como LLaMa 2, Mistral 7B ofrece capacidades similares o mejores pero con menos carga computacional. Mientras que modelos fundacionales como GPT-4 pueden tener un mejor desempeño, Mistral ofrece una \acrshort{api} gratuita a través de HuggingFace. 

El modelo tiene 7B de parámetros y aunque modelos de mas parámetros podrían dar mejores resultados, para la investigación que se desarrolla en este \acrshort{tfg} resulta mas interesante poder realizar las pruebas sin incurrir en costes y tecnología propietaria como la de OpenAI.

Como alternativa a Mistral se ha desarrollado también en la fase de prototipado una segunda versión del Chatbot usando LLaMa 2 de Meta. En general ambos modelos en sus versiones de 7B de parámetros dan resultados similares, pero LLaMa 2 requiere solicitar acceso a la \acrshort{api} para usos en investigación y no se encuentra en la Unión Europea, por lo que no tiene las mismas medidas para la protección de datos.

\section{Fase temprana en la inteligencia artificial generativa}

El inicio de la inteligencia artificial generativa marcó un hito significativo en el campo de la tecnología. Se refiere a la capacidad de las máquinas para generar contenido, especialmente en forma de texto, imágenes y otros tipos de datos. Un momento crucial en este avance fue la introducción de modelos de lenguaje generativos, como \acrfull{gpt}, que son capaces de entender, interpretar y generar texto de manera coherente y contextual.

Este desarrollo ha llevado a avances notables en diversas aplicaciones, como chatbots más inteligentes, asistentes virtuales más avanzados y generación automática de contenido creativo. Desde que OpenAI lanzara ChatGPT el 30 de noviembre de 2022, los avances y mejoras en este campo ha sufrido un avance vertiginoso. 

\imagen{TimelineGAI}{Avances en el campo de la Inteligencia Artificial Generativa en los últimos meses.}{1}

Uno de los factores clave en este rápido desarrollo es el enfoque de preentrenamiento de estos modelos, lo que significa que son entrenados en grandes cantidades de datos antes de ser afinados para tareas específicas. Esto les permite capturar patrones complejos y contextos, lo que resulta en un rendimiento más sofisticado. 

Sin embargo, este avance también ha planteado desafíos éticos y preocupaciones sobre el uso responsable de la inteligencia artificial generativa. La capacidad de crear contenido realista y convincente ha llevado a debates sobre la desinformación, la manipulación de información y la necesidad de salvaguardias para garantizar un uso ético y beneficioso de estas tecnologías.

En resumen, el inicio de la inteligencia artificial generativa es una fase emocionante con avances notables casi a diario, pero también plantea importantes consideraciones éticas y aspectos que se deben consolidar para garantizar un desarrollo tecnológico viable.

\subsection{Fase de investigación y desarrollo vs fase de explotación}

La \acrshort{ia} ha experimentado avances significativos en investigación y desarrollo, con la creación de \acrshort{llm}, redes neuronales profundas y enfoques innovadores para tareas específicas. Sin embargo, la aplicación masiva y generalizada de la \acrshort{ia} a productos en fase de explotación sigue siendo un desafío en muchos casos.

Algunas razones para esta brecha entre la investigación y la implementación amplia incluyen:

\begin{itemize}

\item \textbf{Multitples vias de desarrollo:} A pesar de su corta historia, existen multitud de herramientas, técnicas y estrategias relacionadas con los \acrshort{llm}. En esta fase temprana no se ha consolidado una dirección como paradigma ha seguir por lo que existen muchos caminos de investigación que meses después se abandonan para proseguir por otros mas prometedores.

\item \textbf{Complejidad de Tareas del Mundo Real:} Muchas tareas del mundo real son complejas y requieren un entendimiento profundo del contexto. Aunque los modelos de \acrshort{ia} han progresado, aún pueden enfrentar dificultades para manejar situaciones impredecibles o interpretar información de manera sutil.

\item \textbf{Interpretabilidad y Transparencia:} Los modelos de \acrshort{ia}, especialmente los de aprendizaje profundo, a menudo son cajas negras difíciles de interpretar. En entornos críticos, como la atención médica o la toma de decisiones legales, la interpretabilidad es esencial, y la falta de comprensión completa puede limitar la adopción.

\item \textbf{Ética y Sesgo:} Las preocupaciones éticas relacionadas con el sesgo en los datos y la toma de decisiones algorítmica han generado discusiones importantes. La necesidad de abordar el sesgo y garantizar decisiones justas y equitativas sigue siendo un desafío.

\item \textbf{Regulaciones y Normativas:} La implementación de la \acrshort{ia} a menudo se ve afectada por regulaciones y normativas en evolución. Como la aprobada recientemente en la Unión Europea. Las preocupaciones sobre la privacidad, la seguridad y el impacto social han llevado a la introducción de leyes y estándares que afectan la aplicación generalizada.

\end{itemize}

A pesar de estos desafíos, es importante destacar que la \acrshort{ia} se está utilizando en diversos sectores, desde asistentes virtuales hasta diagnóstico médico asistido por máquina. A medida que la investigación continúa y se abordan los desafíos actuales, es probable que veamos una mayor aplicación de la \acrshort{ia} en productos y servicios en el futuro.

\subsection{Langchain como \textit{framework} de gestión en un momento de cambio constante}

Por lo anteriormente expuesto, se ha optado por usar Langchain como capa de intermedia para el acceso y gestión de los \acrshort{llm}. Existen ventajas he inconvenientes en su uso y no es seguro de que este \textit{framework} sea la estrategia que se imponga al resto de las posibles vías. A continuación se enumeran algunos de los factores que han determinado esta decisión y también algunos de los inconvenientes que ha traído consigo.

Langchain permite, al menos parcialmente, modificar secciones de la aplicación sin tener que modificar completamente la arquitectura del software. Esto es especialmente interesante en un momento en el que se lanzan nuevos \acrshort{llm} y herramientas casi semanalmente. Como se ha indicado anteriormente, la tecnología está actualmente en un momento de cambio constante y de rápido desarrollo, es una fase interesantisima pero también compleja para el desarrollador. 

El uso de esta capa intermedia permite abstraer parte de la implementación, y que cambios de estrategia supongan solo un cambio en una linea de código. Esto funciona bien parcialmente pero no es siempre posible. Por ejemplo la selección del \acrshort{llm} conlleva mas que un simple cambio de argumentos en muchos casos, y no todos los \acrshort{llm} soportan las mismas características o el mismo \textit{prompt}.

Cabe destacar que Langchain se lanzó en Octubre de 2022, por lo que es un \textit{framework} en constante cambio. Muchas funciones se renuevan y son reemplazadas por otras estrategias, haciendo el mantenimiento de código una tarea compleja. Durante la fase de desarrollo, partes del código han tenido que ser reescritas ya que accesos o funciones recomendadas un mes antes, ya no estaban disponibles.

Un aspecto negativo de la elección de esta capa intermedia es la limitación en si misma que este acceso supone. No se tiene control sobre la implementación de las funciones y esto hace que no se pueda optimizar la aplicación fácilmente. A mayores, Langchain está centrada en el uso de la \acrshort{api} de OpenAI, por lo que no todos los \acrshort{llm} reciben el mismo soporte. En el caso de este proyecto usando Mistral 7B, esto supone que parte de las funciones u opciones no esten soportadas.

\section{Preprocesamiento de datos}

El preprocesamiento de datos es una parte fundamental del desarrollo de \acrlong{llm} y \acrlong{rag}. Aunque los detalles específicos del preprocesamiento pueden variar según la tarea y la arquitectura del modelo, algunos pasos comunes incluyen:

\begin{itemize}

\item \textbf{Tokenización:} Los modelos de lenguaje trabajan con unidades más pequeñas llamadas tokens. Tokenizar un texto implica dividirlo en estas unidades, que podrían ser palabras, subpalabras o incluso caracteres.

\item \textbf{Normalización:} Esto implica convertir el texto a un formato estándar, como convertir todas las letras a minúsculas. Esto ayuda a que el modelo no trate las mismas palabras en diferentes formas como entidades separadas.
 
\item \textbf{Eliminación de \textit{Stopwords}:} Para algunos modelos, puede ser beneficioso eliminar palabras comunes que no aportan mucha información (como "y", "o", "el", etc.) para reducir el ruido en los datos.

\item \textbf{Lidiar con Datos No Estructurados:} Si los datos contienen elementos no textuales, como imágenes o tablas, se debe tener un proceso para manejarlos o convertirlos en un formato que el modelo pueda entender.

\item \textbf{Segmentación de Texto:} Para tareas específicas, como la recuperación de respuesta, puede ser útil dividir el texto en segmentos más pequeños para facilitar la búsqueda y la recuperación de información relevante.

\item \textbf{Manejo de Datos Desbalanceados:} Si los datos están desbalanceados (por ejemplo, se tienen muchas más instancias de una clase que de otra), es posible que se desean aplicar técnicas para abordar este desequilibrio.

\end{itemize}

El preprocesamiento puede variar según la tarea y el modelo específico que se está utilizando. Algunos modelos, como \acrfull{gpt}, han demostrado ser bastante robustos y pueden manejar datos en bruto con un preprocesamiento mínimo, mientras que otros modelos pueden requerir una preparación más cuidadosa de los datos. 

Además, para \acrshort{rag}, también hay un enfoque importante en la creación de conjuntos de datos que vinculen preguntas con respuestas relevantes para el entrenamiento efectivo del modelo. 

Los datos que se disponen del \acrshort{faq} para \acrshort{tfg}, la información está en una estructura poco ventajosa para el Chatbot basado en \acrshort{llm}. Se disponen de distintas fuentes de datos que se han de incorporar al \acrshort{rag}.

\subsection{\textit{Document Loaders}}

Los \textit{document loaders} en el contexto de \acrlong{llm} como LLaMa.cpp o Langchain se refieren a componentes o módulos diseñados para cargar documentos o datos en la memoria del modelo. Estos documentos actúan como contexto o información de referencia que el modelo puede utilizar durante el proceso de inferencia para comprender y responder de manera más precisa a las consultas o preguntas que se le presentan. Basicamente es la función que se describe en el método \acrshort{rag}.

En términos generales, la carga de documentos es esencial para proporcionar contexto y conocimiento al modelo, mejorando así su capacidad para generar respuestas significativas. El proceso de carga de documentos implica tomar información desde diversas fuentes, como bases de datos, páginas web, archivos de texto, etc., y convertirla en un formato que el modelo pueda entender y utilizar.

\imagen{DocumentLoaders}{Matriz de representación de los multiples \textit{Document Loaders} disponibles en LangChain.}{1}

En resumen, los \textit{document loaders} son parte fundamental del proceso de preparación de datos para \acrlong{llm}, asegurando que tengan acceso a la información relevante que les permita realizar tareas específicas de manera efectiva.

\subsection{Datos disponibles}

Se han realizado unas pruebas con los datos sin preprocesar y los resultados no han sido buenos. La información se encuentra en formato .docx y contiene tanto comentarios, como tablas como texto en párrafos. Está formato de información estaba creado para el Chatbot que usaba DialogFlow para la creación de la aplicación. 

Esta información es segmentada en el proceso de creación de los \textit{Embeddings} y sin una estructura definida los segmentos no mantenían la estructura semántica. Es cierto que el Chatbot respondía algunas preguntas correctamente al exportar los datos a .txt de forma automática, pero parte de las preguntas y respuestas se mezclaban al no separarse correctamente.

Se han exportado los datos a formato .csv y al usar el \textit{Data Loader} de Langchain se ha especificado como fragmentar la información correctamente y como interpretar cada columna. Esto ha supuesto un salto considerable los resultados del Chatbot y en general la información se recupera adecuadamente de la base de datos vectorial.

Para usar técnicas \acrshort{rag} y en general \acrshort{pln}, se deben usar datos mas similares al lenguaje natural o con una estructura definida. El \acrshort{llm} funciona muy bien en gestionar el lenguaje pero no puede gestionar datos desestructurados, tablas o imágenes.

\section{Validación y pruebas en el procesamiento del lenguaje natural}

La validación de \acrshort{llm} y \acrshort{rag} sigue principios generales de evaluación de modelos de aprendizaje automático. Algunas estrategias comunes son:

\begin{itemize}

\item \textbf{Perplejidad:} La perplejidad es una medida común para evaluar modelos de lenguaje. Cuanto menor sea la perplejidad en un conjunto de datos, mejor es el modelo en la tarea de predecir la secuencia de palabras.

\item \textbf{Evaluación Humana:} Se pueden realizar evaluaciones humanas donde se pide a los evaluadores que califiquen la calidad de las generaciones del modelo en términos de fluidez, coherencia y relevancia.

\item \textbf{\textit{Benchmarks} Estándar:} Utilizar conjuntos de datos de referencia o \textit{benchmarks} estándar puede proporcionar una comparación objetiva con otros modelos.

\item \textbf{Ranking de Respuestas:} Dado que \acrshort{rag} está diseñado para recuperar respuestas de un conjunto de documentos, la evaluación a menudo implica comparar la respuesta generada con respuestas de referencia y clasificarlas según su relevancia.

\item \textbf{BLEU y Otras Métricas de Evaluación de Texto:} Métricas como BLEU se utilizan a menudo para evaluar la similitud entre la respuesta generada y las respuestas de referencia.

\item \textbf{Conjuntos de Datos de Preguntas y Respuestas:} Se pueden crear conjuntos de datos específicos para \acrshort{rag}, donde se proporcionan preguntas y se espera que el modelo recupere respuestas relevantes de documentos externos.

\end{itemize}

Es importante recordar que no hay una métrica única que capture completamente la calidad de un modelo de lenguaje o de respuesta generativa. La combinación de varias métricas y evaluaciones humanas a menudo brinda una visión más completa del rendimiento del modelo. Además, la elección de la estrategia de evaluación puede depender de la tarea específica y de los objetivos del modelo.

De las métricas nombradas anteriormente se ha optado por hacer una mezcla de Evaluación Humana, Conjunto de Datos de Pregunta/Respuesta y \textit{Benchmark}. 

En una primera etapa se ha realizado una validación genérica basada en la evaluación humana. Es relativamente fácil descartar algunas configuraciones que dan respuestas alejadas de lo que se busca.

Un vez que se tiene una estrategia general, se ha realizado un proceso de \textit{Benchmarking} usando una lista de preguntas y respuestas y comparando los resultado con las respuestas previstas. Esto permite realizar un ranking de que configuración da mejores resultados en el \acrshort{rag}. Para ello se han usado 22 preguntas del set de preguntas del que se dispone y se ha generado para cada variación un reporte.

\imagen{Validacion1}{Ejemplo de reporte de testeo de una posible configuración del RAG.}{1}

\subsection{Usar un LLM para validar un LLM}

Un interesante aspecto que se plantea al validar respuestas del chatbot es determinar que es una respuesta correcta. Los \acrshort{llm} son por naturaleza no deterministas y en el lenguaje natural a diferencia de en problemas matematicos, dos respuestas pueden ser distintas y a la vez correctas. 

Para ello se utiliza una interesante estrategia que consiste en usar un \acrshort{llm} para valorar si la respuesta generada por el Chatbot (que ha sido generada por un \acrshort{llm}) contiene la misma información que la respuesta esperada.

Explicado de una forma simplificada, es una llamada a un modelo de generación que incluye un \textit{Prompt} del tipo:

\begin{verbatim}
Prompt: Tienes que valorar si dos respuestas de dadas 
son equivalentes. La información en {respuesta generada} 
es equivalente a la que contiene {respuesta esperada}.
\end{verbatim}

La respuesta de esta consulta será una boleana que nos dirá si es correcto o incorrecto. En teoría esto se puede aplicar usando Langchain pero lamentablemente no está exento de fallos. Esta validación automática es rápida pero tiene un tasa de fallo significativa. Vale como indicación general de lo bueno o malo que es una solución pero se debe comprobar de forma manual.

\imagen{Validacion2}{Ejemplo de la validación de Preguntas y Respuestas del chatbot y su respuesta generada.}{1}
\capitulo{6}{Trabajos relacionados}

\section{Anteriores TFG de Chatbots}

\subsection{UBUAssistant}

Proyecto de Daniel Santidrián Alonso continuado por Carlos González Calatrava en 2018 tutorizado por Pedro Renedo Fernández.

Se trata de un asistente virtual para Android que cuenta con asistente de voz. El objetivo es facilitar las búsquedas en la web de la \acrshort{ubu} al realizarlas desde un smartphone.

La aplicación está alojada en un servidor de Azure que ofrece una suscripción gratuita para estudiantes, aunque con ciertas limitaciones. Utiliza el algoritmo de Razonamiento Basado en Casos~\cite{UBUAssistant}. 

\subsection{Chatbot for Tourist Recommendations}

Proyecto de Jasmin Wellnitz en 2017 tutorizado por el Dr. Bruno Baruque Zanon. 

Chatbot implementado en Telegram actualmente fuera de servicio cuya finalidad era dar recomendaciones turísticas en inglés.
Utilizó la misma tecnología que este proyecto, en aquel entonces se llamaba API.AI y era el mismo \acrshort{pln} que es ahora Dialogflow tras ser comprado en 2016 por Google que optó al año siguiente por cambiarle el nombre.

El proyecto esta alojado en la \textit{Platform as a Service} Heroku. Utiliza la versión de prueba gratuita pero tiene limitaciones en cuanto al uso mensual~\cite{ChatbotTourist}. 

\subsection{UBUVoiceAssistant}

Proyecto de Álvaro Delgado Pascual tutorizado por el Dr. Raúl Marticorena Sanchez.

Se trata de una aplicación que por medio de un asistente de voz permite al usuario obtener información sobre una plataforma de Moodle. Está desarrollado en Python y utiliza el asistente de voz Mycroft~\cite{UBUVoiceAssistant}. 

\subsection{UBU-Chatbot}

Proyecto de Alfredo Asensio Vázquez en 2021 tutorizado por Dr. Raúl Marticorena Sanchez.

Se trata de un Bot conversacional (chatbot) cuyo objetivo es dar respuesta a todas las dudas que los alumnos puedan tener sobre el funcionamiento de la asignatura \acrlong{tfg}. Basado en Dialogflow de Google, estructura las preguntas y respuestas según su intención y hace una recuperación por similitud. Dispone de una versión para la modalidad \textit{online} y otra para presencial. UBU-Chatbot está integrado en la plataforma Moodle de la Universidad y en Slack~\cite{UBU-Chatbot}. 

Este trabajo realizado en 2021 es el precursor de este \acrshort{tfg}. Por ello se usará para realizar una comparativa que sirva para evaluar la nueva solución basada en una tecnología completamente diferente.

\section{Estado del arte de los Chatbots basados en LLMs}

Teachbot
\subsection{Teachbot}

Proyecto de Miguel Collado en 2023 tutorizado por Carlos López Nozal, Ismael Ramos Pérez y Dr. Raúl Marticorena Sanchez.

Un chatbot cuyo objetivo es ayudar a los estudiantes del Prácticum de Formación de Profesores de Lenguas Extranjeras de la \acrlong{ubu}, en el desarrollo del mismo. Se parte de un documento guía que indica a los estudiantes como realizar su Practicum y su implementación en una plantilla Trello. Este chatbot será implementado en diferentes plataformas. El lenguaje conversacional de modelado era en inglés.~\cite{Teachbot}. 

Este trabajo realizado en 2023 ya usa la \acrshort{api} de OpenAI para enriquecer las respuestas del Chatbot de DialogFlow.

\section{Estado del arte de los Chatbots basados en LLMs}

\begin{itemize}
    \item \textbf{ChatGPT de OpenAI:} OpenAI ha liderado el desarrollo de modelos de lenguaje avanzados, como ChatGPT basado en GPT-3. Este chatbot demostró una comprensión contextual impresionante y una capacidad para generar respuestas coherente en diversas situaciones. La implementación de modelos como ChatGPT marcó un hito en la creación de chatbot conversacionales más sofisticados.

    \item \textbf{BlenderBot 2.0 de Facebook AI:} Facebook AI desarrolló BlenderBot 2.0, un chatbot que utiliza una combinación de \acrshort{llm} y enfoques de aprendizaje por refuerzo. Este proyecto destacó por su capacidad para sostener conversaciones más largas y coherentes, mostrando avances en la comprensión contextual.

    \item \textbf{DALL-E de OpenAI:} Aunque no es un chatbot en el sentido tradicional, DALL-E es un modelo generativo de imágenes desarrollado por OpenAI utilizando una arquitectura similar a la de los \acrshort{llm}. Este proyecto ilustra la versatilidad de los modelos generativos en la creación de contenido multimedia.

    \item \textbf{\acrshort{rag} for Conversational AI de Facebook AI:} Facebook AI introdujo \acrshort{rag} para mejorar la recuperación de respuestas en chatbots, integrando conocimientos de documentos externos. La implementación de \acrshort{rag} aborda la mejora de la relevancia y diversidad de las respuestas en entornos conversacionales.

    \item \textbf{Conversaciones Más Naturales con \acrshort{llm} de Google Research:} Proyectos de Google Research han explorado cómo los \acrshort{llm}, como BERT, mejoran la naturalidad y relevancia de las respuestas en chatbots. Este trabajo destaca la importancia de la comprensión contextual para lograr conversaciones más fluidas y significativas.

\end{itemize}

\section{Comparativa con UBU-Chatbot}

Esta comparativa busca proporcionar una evaluación de los chatbots, permitiendo una toma de decisiones sobre la idoneidad o no adopción de esta nueva tecnología. Para comparar y validar de las respuestas se pueden agrupar las preguntas en tres dimensiones:

\begin{itemize}

\item Preguntas de tutoría administrativas: \acrshort{faq} recurso de información estable y particular de \acrshort{tfg} en la \acrshort{ubu}.

\item Preguntas de tutoría generales: conceptos de planificación de proyectos~\cite{digit}.

\item Preguntas de tutoría específicas de un \acrshort{tfg}: Por ejemplo, ¿cómo documentar el funcionamiento del chatbot en una memoria de \acrshort{tfg}?.

\end{itemize}

\subsection{Preguntas de tutoría administrativas}

Para poder comparar ambos chatbots se han realizado las mismas preguntas en ambos chatbots y se han comprobado los resultados.

Primero se comprueban la respuesta de chatbot basado en DialogFlow. Después de un saludo, el chatbot repite el mensaje genérico de inicio y no lo adapta al saludo, ver figura~\ref{fig:ubu-chatbot1}.

\imagen{ubu-chatbot1}{Ejemplo de pregunta de tutoría administrativa, saludo, del chatbot de DialogFlow.}{0.5}

Se ve una clara mejora en el chatbot que se ha realizado en este \acrshort{tfg}. El chatbot adapta el mensaje de respuesta al saludo del usuario. Así se responde dando los buenos días, o las buenas tardes o simplemente hola, dependiendo de como haya saludado el usuario, ver figura~\ref{fig:chatbot1}.

\imagen{chatbot1}{Ejemplo de pregunta de tutoría administrativa, saludo, del chatbot basado en LLM y RAG.}{1}

Otra pregunta administrativa básica es la composición del tribunal del \acrshort{tfg}. El chatbot existente responde correctamente a las preguntas sobre la composición del tribunal, ver figura~\ref{fig:ubu-chatbot2}.

\imagen{ubu-chatbot2}{Ejemplo de pregunta de tutoría administrativa, profesores del tribunal, del chatbot de DialogFlow.}{0.5}

De igual manera este chatbot basado en \acrshort{llm}, responde a las preguntas sobre la composición del tribunal, ver figura~\ref{fig:chatbot2}.

\imagen{chatbot2}{Ejemplo de pregunta de tutoría administrativa, profesores del tribunal, del chatbot basado en LLM y RAG.}{1}

Sin embargo, el chatbot anterior no gestiona bien la formulaciones complejas de la pregunta, como se ve en la figura~\ref{fig:ubu-chatbot2a}.

\imagen{ubu-chatbot2a}{Ejemplo de pregunta de tutoría administrativa, profesores del tribunal pregunta compleja, del chatbot de DialogFlow.}{0.5}

Como era de esperar el chatbot realizado en este \acrshort{tfg} gestiona mejor este tipo de preguntas. Los \acrshort{llm} son \acrlong{pln} y son capaces de interpretar mejor la semántica de las preguntas, ver figura~\ref{fig:chatbot2a}.

\imagen{chatbot2a}{Ejemplo de pregunta de tutoría administrativa, profesores del tribunal pregunta compleja, del chatbot basado en LLM y RAG.}{1}

\subsection{Preguntas de tutoría generales e históricos}

El chatbot basado en DialogFlow no incluía información relativa a los históricos del \acrshort{tfg}. Ver figura~\ref{fig:ubu-chatbot3}.

\imagen{ubu-chatbot3}{Ejemplo de pregunta de tutoría general del chatbot basado en LLM y RAG.}{0.5}

Esta mejora ha sido una de las mas importantes de este chatbot, en el que se ha incluido mas información en el entrenamiento del \acrshort{rag}. Se puede ver en la figura~\ref{fig:chatbot3}, que se puede proporcionar información relativa al histórico de \acrshort{tfg}.

\imagen{chatbot3}{Ejemplo de pregunta de tutoría general del chatbot basado en LLM y RAG.}{1}

\subsection{Preguntas de tutoría específicas de un TFG}

Este tipo de preguntas son las mas difíciles de responder para un chatbot. Se requiere un procesamiento semántico mas complejo y están ligados a una tarea muy especifica. Como se ve en la figura~\ref{fig:ubu-chatbot4}, el chatbot basado en DialogFlow ni siquiera entiende la pregunta y responde algo más genérico.

\imagen{ubu-chatbot4}{Ejemplo de pregunta de tutoría general del chatbot basado en LLM y RAG.}{0.5}

Sin embargo este chatbot hace uso del \acrshort{llm} para dar una respuesta coherente. En este caso la respuesta sería perfectamente válida, ver figura~\ref{fig:chatbot4}. Aunque le falte algo de profundidad a la respuesta, sea un poco genérica y no pueda sustituir la labor de un tutor, es un muy buen comienzo.

\imagen{chatbot4}{Ejemplo de pregunta de tutoría especificas del chatbot basado en LLM y RAG.}{1}

\capitulo{7}{Conclusiones y Líneas de trabajo futuras}

Todo proyecto debe incluir las conclusiones que se derivan de su desarrollo. Éstas pueden ser de diferente índole, dependiendo de la tipología del proyecto, pero normalmente van a estar presentes un conjunto de conclusiones relacionadas con los resultados del proyecto y un conjunto de conclusiones técnicas. 
Además, resulta muy útil realizar un informe crítico indicando cómo se puede mejorar el proyecto, o cómo se puede continuar trabajando en la línea del proyecto realizado. 

\section{El futuro de los LLM }

Aunque ChatGPT es la última novedad, no es más que un pequeño paso hacia lo que está por venir en el ámbito de los \acrshort{llm}. Aunque no se puede predecir el futuro, hay algunas tendencias que marcarán el camino de la innovación en los próximos anos\cite{Tolaka}.

\begin{enumerate}
    
\item Modelos autónomos que se mejoran a sí mismos: Es probable que estos \acrshort{llm} tengan la capacidad de generar datos de entrenamiento para mejorar su propio rendimiento. Esto puede ser especialmente útil una vez agotadas las ingentes cantidades de información disponibles en Internet. Como ejemplo reciente, el \acrshort{llm} de Google fue capaz de generar sus propias preguntas y respuestas y ajustarse en consecuencia.

\item Modelos capaces de verificar sus propios resultados: Los \acrshort{llm} que pueden proporcionar fuentes para la información que generan pueden dar mayor credibilidad a la tecnología en su conjunto. Por ejemplo, WebGPT de OpenAI es capaz de generar respuestas precisas y detalladas con fuentes de respaldo.

\item El desarrollo de modelos expertos dispersos: Los \acrshort{llm} más reconocidos de la actualidad tienen varias características en común: son modelos densos, autosupervisados y preentrenados basados en la arquitectura de \textit{transformaers}. Sin embargo, los modelos expertos dispersos están llevando la tecnología en otra dirección. Con estos modelos sólo es necesario activar los parámetros relevantes, lo que los hace más grandes y complejos. Al mismo tiempo, requieren menos recursos y consumo de energía para el entrenamiento del modelo.

\end{enumerate}

\printnoidxglossary[type=\acronymtype]

\bibliographystyle{plain}
\bibliography{bibliografia}

\end{document}
