\capitulo{6}{Trabajos relacionados}

\section{Anteriores TFG de Chatbots}

\subsection{UBUAssistant}

Proyecto de Daniel Santidrián Alonso continuado por Carlos González Calatrava en 2018 tutorizado por Pedro Renedo Fernández.

Se trata de un asistente virtual para Android que cuenta con asistente de voz. El objetivo es facilitar las búsquedas en la web de la \acrshort{ubu} al realizarlas desde un smartphone.

La aplicación está alojada en un servidor de Azure que ofrece una suscripción gratuita para estudiantes, aunque con ciertas limitaciones. Utiliza el algoritmo de Razonamiento Basado en Casos~\cite{UBUAssistant}. 

\subsection{Chatbot for Tourist Recommendations}

Proyecto de Jasmin Wellnitz en 2017 tutorizado por el Dr. Bruno Baruque Zanon. 

Chatbot implementado en Telegram actualmente fuera de servicio cuya finalidad era dar recomendaciones turísticas en inglés.
Utilizó la misma tecnología que este proyecto, en aquel entonces se llamaba API.AI y era el mismo \acrshort{pln} que es ahora Dialogflow tras ser comprado en 2016 por Google que optó al año siguiente por cambiarle el nombre.

El proyecto esta alojado en la \textit{Platform as a Service} Heroku. Utiliza la versión de prueba gratuita pero tiene limitaciones en cuanto al uso mensual~\cite{ChatbotTourist}. 

\subsection{UBUVoiceAssistant}

Proyecto de Álvaro Delgado Pascual tutorizado por el Dr. Raúl Marticorena Sanchez.

Se trata de una aplicación que por medio de un asistente de voz permite al usuario obtener información sobre una plataforma de Moodle. Está desarrollado en Python y utiliza el asistente de voz Mycroft~\cite{UBUVoiceAssistant}. 

\subsection{UBU-Chatbot}

Proyecto de Alfredo Asensio Vázquez en 2021 tutorizado por Dr. Raúl Marticorena Sanchez.

Se trata de un Bot conversacional (chatbot) cuyo objetivo es dar respuesta a todas las dudas que los alumnos puedan tener sobre el funcionamiento de la asignatura \acrlong{tfg}. Basado en Dialogflow de Google, estructura las preguntas y respuestas según su intención y hace una recuperación por similitud. Dispone de una versión para la modalidad \textit{online} y otra para presencial. UBU-Chatbot está integrado en la plataforma Moodle de la Universidad y en Slack~\cite{UBU-Chatbot}. 

Este trabajo realizado en 2021 es el precursor de este \acrshort{tfg}. Por ello se usará para realizar una comparativa que sirva para evaluar la nueva solución basada en una tecnología completamente diferente.

\section{Estado del arte de los Chatbots basados en LLMs}

Teachbot
\subsection{Teachbot}

Proyecto de Miguel Collado en 2023 tutorizado por Carlos López Nozal, Ismael Ramos Pérez y Dr. Raúl Marticorena Sanchez.

Un chatbot cuyo objetivo es ayudar a los estudiantes del Prácticum de Formación de Profesores de Lenguas Extranjeras de la \acrlong{ubu}, en el desarrollo del mismo. Se parte de un documento guía que indica a los estudiantes como realizar su Practicum y su implementación en una plantilla Trello. Este chatbot será implementado en diferentes plataformas. El lenguaje conversacional de modelado era en inglés.~\cite{Teachbot}. 

Este trabajo realizado en 2023 ya usa la \acrshort{api} de OpenAI para enriquecer las respuestas del Chatbot de DialogFlow.

\section{Estado del arte de los Chatbots basados en LLMs}

\begin{itemize}
    \item \textbf{ChatGPT de OpenAI:} OpenAI ha liderado el desarrollo de modelos de lenguaje avanzados, como ChatGPT basado en GPT-3. Este chatbot demostró una comprensión contextual impresionante y una capacidad para generar respuestas coherente en diversas situaciones. La implementación de modelos como ChatGPT marcó un hito en la creación de chatbot conversacionales más sofisticados.

    \item \textbf{BlenderBot 2.0 de Facebook AI:} Facebook AI desarrolló BlenderBot 2.0, un chatbot que utiliza una combinación de \acrshort{llm} y enfoques de aprendizaje por refuerzo. Este proyecto destacó por su capacidad para sostener conversaciones más largas y coherentes, mostrando avances en la comprensión contextual.

    \item \textbf{DALL-E de OpenAI:} Aunque no es un chatbot en el sentido tradicional, DALL-E es un modelo generativo de imágenes desarrollado por OpenAI utilizando una arquitectura similar a la de los \acrshort{llm}. Este proyecto ilustra la versatilidad de los modelos generativos en la creación de contenido multimedia.

    \item \textbf{\acrshort{rag} for Conversational AI de Facebook AI:} Facebook AI introdujo \acrshort{rag} para mejorar la recuperación de respuestas en chatbots, integrando conocimientos de documentos externos. La implementación de \acrshort{rag} aborda la mejora de la relevancia y diversidad de las respuestas en entornos conversacionales.

    \item \textbf{Conversaciones Más Naturales con \acrshort{llm} de Google Research:} Proyectos de Google Research han explorado cómo los \acrshort{llm}, como BERT, mejoran la naturalidad y relevancia de las respuestas en chatbots. Este trabajo destaca la importancia de la comprensión contextual para lograr conversaciones más fluidas y significativas.

\end{itemize}

\section{Comparativa con UBU-Chatbot}

Esta comparativa busca proporcionar una evaluación de los chatbots, permitiendo una toma de decisiones sobre la idoneidad o no adopción de esta nueva tecnología. Para comparar y validar de las respuestas se pueden agrupar las preguntas en tres dimensiones:

\begin{itemize}

\item Preguntas de tutoría administrativas: \acrshort{faq} recurso de información estable y particular de \acrshort{tfg} en la \acrshort{ubu}.

\item Preguntas de tutoría generales: conceptos de planificación de proyectos~\cite{digit}.

\item Preguntas de tutoría específicas de un \acrshort{tfg}: Por ejemplo, ¿cómo documentar el funcionamiento del chatbot en una memoria de \acrshort{tfg}?.

\end{itemize}

\subsection{Preguntas de tutoría administrativas}

Para poder comparar ambos chatbots se han realizado las mismas preguntas en ambos chatbots y se han comprobado los resultados.

Primero se comprueban la respuesta de chatbot basado en DialogFlow. Después de un saludo, el chatbot repite el mensaje genérico de inicio y no lo adapta al saludo, ver figura~\ref{fig:ubu-chatbot1}.

\imagen{ubu-chatbot1}{Ejemplo de pregunta de tutoría administrativa, saludo, del chatbot de DialogFlow.}{0.5}

Se ve una clara mejora en el chatbot que se ha realizado en este \acrshort{tfg}. El chatbot adapta el mensaje de respuesta al saludo del usuario. Así se responde dando los buenos días, o las buenas tardes o simplemente hola, dependiendo de como haya saludado el usuario, ver figura~\ref{fig:chatbot1}.

\imagen{chatbot1}{Ejemplo de pregunta de tutoría administrativa, saludo, del chatbot basado en LLM y RAG.}{1}

Otra pregunta administrativa básica es la composición del tribunal del \acrshort{tfg}. El chatbot existente responde correctamente a las preguntas sobre la composición del tribunal, ver figura~\ref{fig:ubu-chatbot2}.

\imagen{ubu-chatbot2}{Ejemplo de pregunta de tutoría administrativa, profesores del tribunal, del chatbot de DialogFlow.}{0.5}

De igual manera este chatbot basado en \acrshort{llm}, responde a las preguntas sobre la composición del tribunal, ver figura~\ref{fig:chatbot2}.

\imagen{chatbot2}{Ejemplo de pregunta de tutoría administrativa, profesores del tribunal, del chatbot basado en LLM y RAG.}{1}

Sin embargo, el chatbot anterior no gestiona bien la formulaciones complejas de la pregunta, como se ve en la figura~\ref{fig:ubu-chatbot2a}.

\imagen{ubu-chatbot2a}{Ejemplo de pregunta de tutoría administrativa, profesores del tribunal pregunta compleja, del chatbot de DialogFlow.}{0.5}

Como era de esperar el chatbot realizado en este \acrshort{tfg} gestiona mejor este tipo de preguntas. Los \acrshort{llm} son \acrlong{pln} y son capaces de interpretar mejor la semántica de las preguntas, ver figura~\ref{fig:chatbot2a}.

\imagen{chatbot2a}{Ejemplo de pregunta de tutoría administrativa, profesores del tribunal pregunta compleja, del chatbot basado en LLM y RAG.}{1}

\subsection{Preguntas de tutoría generales e históricos}

El chatbot basado en DialogFlow no incluía información relativa a los históricos del \acrshort{tfg}. Ver figura~\ref{fig:ubu-chatbot3}.

\imagen{ubu-chatbot3}{Ejemplo de pregunta de tutoría general del chatbot basado en LLM y RAG.}{0.5}

Esta mejora ha sido una de las mas importantes de este chatbot, en el que se ha incluido mas información en el entrenamiento del \acrshort{rag}. Se puede ver en la figura~\ref{fig:chatbot3}, que se puede proporcionar información relativa al histórico de \acrshort{tfg}.

\imagen{chatbot3}{Ejemplo de pregunta de tutoría general del chatbot basado en LLM y RAG.}{1}

\subsection{Preguntas de tutoría específicas de un TFG}

Este tipo de preguntas son las mas difíciles de responder para un chatbot. Se requiere un procesamiento semántico mas complejo y están ligados a una tarea muy especifica. Como se ve en la figura~\ref{fig:ubu-chatbot4}, el chatbot basado en DialogFlow ni siquiera entiende la pregunta y responde algo más genérico.

\imagen{ubu-chatbot4}{Ejemplo de pregunta de tutoría general del chatbot basado en LLM y RAG.}{0.5}

Sin embargo este chatbot hace uso del \acrshort{llm} para dar una respuesta coherente. En este caso la respuesta sería perfectamente válida, ver figura~\ref{fig:chatbot4}. Aunque le falte algo de profundidad a la respuesta, sea un poco genérica y no pueda sustituir la labor de un tutor, es un muy buen comienzo.

\imagen{chatbot4}{Ejemplo de pregunta de tutoría especificas del chatbot basado en LLM y RAG.}{1}
