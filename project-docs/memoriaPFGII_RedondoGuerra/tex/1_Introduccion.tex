\capitulo{1}{Introducción}

Como parte del grado en Ingeniería Informática, es necesaria la realización de un Trabajo de ``Fin de Grado'' (TFG) que marca el cierre de la etapa de formación. Sin embargo, este proyecto no está exento de desafíos, y los estudiantes a menudo se enfrentan a una multitud de preguntas y obstáculos, desde la selección del tema adecuado hasta la presentación final del proyecto\cite{lopez2010final}. La complejidad de este proceso, combinada con la necesidad de optimizar los tiempos y la orientación, ha dado lugar a una necesidad de herramientas y recursos que asistan a los estudiantes en esta parte de su etapa académica\cite{lopez2009proceso}.

En respuesta a esta demanda, este proyecto se centra en el desarrollo de un asistente virtual innovador: un chatbot basado en ``\textit{Large Language Models}'' (LLM) que tiene como objetivo proporcionar respuestas precisas y adaptadas y orientación en tiempo real a los estudiantes que comienzan a realizar el TFG en Ingeniería Informática\cite{Lewis2020}.

La base de este chatbot reside en su capacidad para comprender y procesar preguntas complejas, así como en su habilidad para generar respuestas coherentes y relevantes. Para mejorar aún más su eficacia y precisión, se implementaran técnicas de ``\textit{Retrieval-augmented Generation}'' (RAG) que permitirán enriquecer las respuestas del chatbot al aprovechar la información contenida en documentos existentes sobre TFGs. Información que será debidamente embebida en vectores(en el espacio del modelo) y almacenada en una base de datos vectorizada. Este enfoque combina la potencia de los LLM con la precisión de la recuperación de información RAG, creando así un asistente virtual altamente competente para este caso de uso.

Además, en aras de evaluar y verificar la eficacia de este nuevo chatbot, se llevará a cabo una comparación con un chatbot preexistente desarrollado en un proyecto de fin de grado anterior. Este análisis nos permitirá identificar las mejoras y ventajas de nuestra nueva aproximación.

En resumen, este proyecto aborda la problemática de acceder a una gran cantidad de información que no es de dominio general a través de la combinación de LLM y técnicas RAG. Se espera no solo crear un chatbot que pueda ser utilizado para este caso de uso sino también generar un proceso que pueda ser aplicado a otros casos de uso similares en los que exista una gran cantidad de información especifica y que no sea de dominio público. 




