\newpage
\apendice{Especificación de Requisitos}

\section{Introducción}

En esta sección, se exponen los objetivos generales del proyecto junto con los requisitos y su correspondiente especificación. Se realiza un análisis exhaustivo tanto de los requisitos funcionales como de los no funcionales.

\begin{itemize}

\item \textbf{Requisitos funcionales:} Estos se refieren a los comportamientos específicos que el sistema debe exhibir y están directamente vinculados con los casos de uso.

\item \textbf{Requisitos no funcionales:} En esta categoría se detallan los criterios, restricciones y condiciones que el cliente impone al proyecto, estableciendo pautas que no están directamente relacionadas con comportamientos específicos, sino con características más amplias del sistema.
\end{itemize}

\section{Objetivos generales}

\begin{itemize}

\item Diseño y desarrollo del chatbot basado en \acrshort{llm}: El primer objetivo es diseñar y desarrollar un chatbot que utilice \acrlong{llm} para responder a las preguntas y dudas de los estudiantes en relación con la realización de sus \acrlong{tfg} en Ingeniería Informática.
    
\item Integración de técnicas de \acrshort{rag} para mejora de la precisión: El segundo objetivo es integrar técnicas de \acrlong{rag} en el chatbot. Estas técnicas se utilizarán para mejorar la precisión y relevancia de las respuestas del chatbot, aprovechando la información contenida en documentos \acrshort{faq}, histórico y reglamento del \acrshort{tfg}.

\item Comparación de rendimiento con chatbot preexistente: El último objetivo implica comparar el rendimiento y la eficacia del chatbot desarrollado en este proyecto con oun chatbot preexistente que fue creado en un proyecto de fin de grado anterior. Esto permitirá identificar las mejoras y ventajas de la nueva aproximación.

\end{itemize}

\section{Catálogo de requisitos}

\subsection{Requisitos funcionales}

\begin{itemize}
\item \textbf{RF-1 Interacción texual:} la aplicación debe permitir al usuario interactuar con ella mediante la introducción de texto.
\item \textbf{RF-2 Procesamiento del lenguaje natural:} la aplicación debe ser capaz de reconocer las preguntas que introduce el usuario y procesarlas para encontrar información relevante.
\item \textbf{RF-3 Formulación de respuestas:} el chatbot ha de ser capaz de formular respuestas en lenguaje natural con información relevante a la pregunta.
\item \textbf{RF-4 Información mediante hipervínculos:} la aplicación debe adjuntar como hipervínculos las redirecciones a otras páginas.
\item \textbf{RF-5 Informe de error:} la aplicación debe informar de un error al usuario cuando su mensaje de entrada no sea un texto válido o no haya sido posible interpretarlo.
\item \textbf{RF-6 Iniciar y cerrar la interfaz conversacional:} la aplicación debe permitir iniciar la interfaz conversacional.
\item \textbf{RF-7 Entrenar al chatbot:} se puede actualizar los datos del bot para que se mejoren las respuestas con nuevos datos.
	

\end{itemize}

\subsection{Requisitos no funcionales}

\begin{itemize}
\item \textbf{RNF-1 Rendimiento:} la aplicación tiene que tener un tiempo de respuesta bajo.  
\item \textbf{RNF-2 Usabilidad:} la aplicación debe ser intuitiva y fácil de entender y utilizar.  
\item \textbf{RNF-3 Imagen corporativa:} la aplicación debe mantener los colores y estética corporativos de la \acrshort{ubu}.  
\item \textbf{RNF-4 Disponibilidad:} la aplicación debe estar disponible el mayor tiempo posible.
\item \textbf{RNF-5 Mantenibilidad:} la aplicación debe ser fácilmente modificable.
\item \textbf{RNF-6 Portabilidad:} la aplicación debe poder ejecutarse en distintas plataformas.
\item \textbf{RNF-7 Compatibilidad de navegadores:} la aplicación debe ser compatible para los navegadores más importantes.
\end{itemize}

\newpage
\section{Especificación de requisitos}

\subsection{Diagrama de casos de uso}

\imagen{diagramaCasosDeUso}{Diagrama de casos de uso del chatbot.}

El actor ``usuario'' va a ser el estudiante, profesor o cualquier persona con acceso a la asignatura que comience una nueva sesión con el chatbot. El actor ``administrador'' es el encargado de actualizar los datos para el \acrshort{rag}, ver figura~\ref{fig:diagramaCasosDeUso}.

\newpage
\subsection{Casos de uso}

\begin{longtable}[H]{@{}ll@{}}
	\toprule
	\begin{minipage}[b]{0.23\columnwidth}\raggedright\strut
		\textbf{CU-01}\strut
	\end{minipage} & \begin{minipage}[b]{0.71\columnwidth}\raggedright\strut
		\textbf{Generación de respuestas.}\strut
	\end{minipage}\tabularnewline
	\midrule
	\endhead  
	\begin{minipage}[t]{0.23\columnwidth}\raggedright\strut
		\textbf{Requisitos asociados}\strut
	\end{minipage} & \begin{minipage}[t]{0.71\columnwidth}\raggedright\strut
		RF-1, RF-2, RF-3, RF-4, RF-5\strut
	\end{minipage}\tabularnewline
	\begin{minipage}[t]{0.23\columnwidth}\raggedright\strut
		\textbf{Descripción}\strut
	\end{minipage} & \begin{minipage}[t]{0.71\columnwidth}\raggedright\strut
		El usuario introduce de manera textual una pregunta al chatbot.\strut
	\end{minipage}\tabularnewline
	\begin{minipage}[t]{0.23\columnwidth}\raggedright\strut
		\textbf{Precondición}\strut
	\end{minipage} & \begin{minipage}[t]{0.71\columnwidth}\raggedright\strut
		El chatbot está iniciado.\strut
	\end{minipage}\tabularnewline
	\begin{minipage}[t]{0.23\columnwidth}\raggedright\strut
		\textbf{Acciones}\strut
	\end{minipage} & \begin{minipage}[t]{0.71\columnwidth}\raggedright\strut
		\begin{enumerate}
			\def\labelenumi{\arabic{enumi}.}
			\tightlist
			\item El programa analiza la pregunta introducida y recupera información relevante.
                \item Se genera una respuesta en lenguaje natural usando la pregunta y la información relevante recuperada.
			\item Se responde al usuario en modo texto y por medio de la interfaz conversacional a su pregunta. 
			\item
			El chatbot queda a la espera de recibir nuevas preguntas.
		\end{enumerate}\strut
	\end{minipage}\tabularnewline
	\begin{minipage}[t]{0.23\columnwidth}\raggedright\strut
		\textbf{Postcondición}\strut
	\end{minipage} & \begin{minipage}[t]{0.71\columnwidth}\raggedright\strut
		Se devuelve un mensaje con la respuesta a la pregunta.\strut
	\end{minipage}\tabularnewline
	\begin{minipage}[t]{0.23\columnwidth}\raggedright\strut
		\textbf{Excepciones}\strut
	\end{minipage} & \begin{minipage}[t]{0.71\columnwidth}\raggedright\strut
		Si el chatbot no es capaz de reconocer la pregunta introducida por el usuario o está mal formulada se informa al usuario por medio de un mensaje de que no ha sido posible entenderle.\strut
	\end{minipage}\tabularnewline
	\begin{minipage}[t]{0.23\columnwidth}\raggedright\strut
		\textbf{Importancia}\strut
	\end{minipage} & \begin{minipage}[t]{0.71\columnwidth}\raggedright\strut
		Alta\strut
	\end{minipage}\tabularnewline
	\bottomrule
	\caption{CU-01 Generación de respuestas.}
\end{longtable}

\newpage
\begin{longtable}[H]{@{}ll@{}}
	\toprule
	\begin{minipage}[b]{0.23\columnwidth}\raggedright\strut
		\textbf{CU-02}\strut
	\end{minipage} & \begin{minipage}[b]{0.71\columnwidth}\raggedright\strut
		\textbf{Actualizar datos del bot.}\strut
	\end{minipage}\tabularnewline
	\midrule
	\endhead  
	\begin{minipage}[t]{0.23\columnwidth}\raggedright\strut
		\textbf{Requisitos asociados}\strut
	\end{minipage} & \begin{minipage}[t]{0.71\columnwidth}\raggedright\strut
		RF-2, RF-7\strut
	\end{minipage}\tabularnewline
	\begin{minipage}[t]{0.23\columnwidth}\raggedright\strut
		\textbf{Descripción}\strut
	\end{minipage} & \begin{minipage}[t]{0.71\columnwidth}\raggedright\strut
		El administrador actualiza los datos del chatbot para que la información sea mas relevante.\strut
	\end{minipage}\tabularnewline
	\begin{minipage}[t]{0.23\columnwidth}\raggedright\strut
		\textbf{Precondición}\strut
	\end{minipage} & \begin{minipage}[t]{0.71\columnwidth}\raggedright\strut
		Se han actualizado los datos de entrenamiento en la carpeta correspondiente.\strut
	\end{minipage}\tabularnewline
	\begin{minipage}[t]{0.23\columnwidth}\raggedright\strut
		\textbf{Acciones}\strut
	\end{minipage} & \begin{minipage}[t]{0.71\columnwidth}\raggedright\strut
		\begin{enumerate}
			\def\labelenumi{\arabic{enumi}.}
			\tightlist
			\item Por línea de comandos se ejecuta el programa de actualización de datos de entrenamiento.
                \item El programa genera una nueva base de datos con la información vectorizada de los documentos \acrshort{faq}, histórico y reglamento del \acrshort{tfg}.
			\item Se guarda la nueva base de datos reemplazando a la anterior existente.
		\end{enumerate}\strut
	\end{minipage}\tabularnewline
	\begin{minipage}[t]{0.23\columnwidth}\raggedright\strut
		\textbf{Postcondición}\strut
	\end{minipage} & \begin{minipage}[t]{0.71\columnwidth}\raggedright\strut
		Se devuelve un mensaje de éxito.\strut
	\end{minipage}\tabularnewline
	\begin{minipage}[t]{0.23\columnwidth}\raggedright\strut
		\textbf{Excepciones}\strut
	\end{minipage} & \begin{minipage}[t]{0.71\columnwidth}\raggedright\strut
		Si el programa no es capaz de actualizar la base de datos, se muestra un mensaje de error.\strut
	\end{minipage}\tabularnewline
	\begin{minipage}[t]{0.23\columnwidth}\raggedright\strut
		\textbf{Importancia}\strut
	\end{minipage} & \begin{minipage}[t]{0.71\columnwidth}\raggedright\strut
		Alta\strut
	\end{minipage}\tabularnewline
	\bottomrule
	\caption{CU-02 Actualizar datos del bot.}
\end{longtable}

\newpage
\begin{longtable}[H]{@{}ll@{}}
	\toprule
	\begin{minipage}[b]{0.23\columnwidth}\raggedright\strut
		\textbf{CU-03}\strut
	\end{minipage} & \begin{minipage}[b]{0.71\columnwidth}\raggedright\strut
		\textbf{Validación del chatbot.}\strut
	\end{minipage}\tabularnewline
	\midrule
	\endhead  
	\begin{minipage}[t]{0.23\columnwidth}\raggedright\strut
		\textbf{Requisitos asociados}\strut
	\end{minipage} & \begin{minipage}[t]{0.71\columnwidth}\raggedright\strut
		RF-1, RF-2, RF-3, RF-4, RF-5, RF-6\strut
	\end{minipage}\tabularnewline
	\begin{minipage}[t]{0.23\columnwidth}\raggedright\strut
		\textbf{Descripción}\strut
	\end{minipage} & \begin{minipage}[t]{0.71\columnwidth}\raggedright\strut
		El administrador verifica que el funcionamiento del chatbot es el adecuado y que las respuestas son parecidas a la información con la que ha sido entrenado.\strut
	\end{minipage}\tabularnewline
	\begin{minipage}[t]{0.23\columnwidth}\raggedright\strut
		\textbf{Precondición}\strut
	\end{minipage} & \begin{minipage}[t]{0.71\columnwidth}\raggedright\strut
		Se ha actualizado la base de datos y la lista de preguntas a verificar.\strut
	\end{minipage}\tabularnewline
	\begin{minipage}[t]{0.23\columnwidth}\raggedright\strut
		\textbf{Acciones}\strut
	\end{minipage} & \begin{minipage}[t]{0.71\columnwidth}\raggedright\strut
		\begin{enumerate}
			\def\labelenumi{\arabic{enumi}.}
			\tightlist
			\item Por línea de comandos se ejecuta el programa de validación del chatbot..
                \item El programa ejecuta una serie de preguntas al chatbot sin iniciar la interfaz gráfica y compara la respuesta obtenida con la esperada.
			\item Se guarda en un archivo la configuración actual, la pregunta realizada y las respuestas obtenidas y esperadas.
		\end{enumerate}\strut
	\end{minipage}\tabularnewline
	\begin{minipage}[t]{0.23\columnwidth}\raggedright\strut
		\textbf{Postcondición}\strut
	\end{minipage} & \begin{minipage}[t]{0.71\columnwidth}\raggedright\strut
		Se ha cerrado correctamente el programa y se ha generado el archivo con la información de la prueba.\strut
	\end{minipage}\tabularnewline
	\begin{minipage}[t]{0.23\columnwidth}\raggedright\strut
		\textbf{Excepciones}\strut
	\end{minipage} & \begin{minipage}[t]{0.71\columnwidth}\raggedright\strut
		Si el programa no es capaz de finalizar el programa de validación, se muestra un mensaje de error.\strut
	\end{minipage}\tabularnewline
	\begin{minipage}[t]{0.23\columnwidth}\raggedright\strut
		\textbf{Importancia}\strut
	\end{minipage} & \begin{minipage}[t]{0.71\columnwidth}\raggedright\strut
		Alta\strut
	\end{minipage}\tabularnewline
	\bottomrule
	\caption{CU-03 Validación del chatbot.}
\end{longtable}