\capitulo{4}{Técnicas y herramientas}

Esta parte de la memoria tiene como objetivo presentar las técnicas metodológicas y las herramientas de desarrollo que se han utilizado para llevar a cabo el proyecto. Si se han estudiado diferentes alternativas de metodologías, herramientas, bibliotecas se puede hacer un resumen de los aspectos más destacados de cada alternativa, incluyendo comparativas entre las distintas opciones y una justificación de las elecciones realizadas. 
No se pretende que este apartado se convierta en un capítulo de un libro dedicado a cada una de las alternativas, sino comentar los aspectos más destacados de cada opción, con un repaso somero a los fundamentos esenciales y referencias bibliográficas para que el lector pueda ampliar su conocimiento sobre el tema.


\section{Retrieval-augmented Generation}

\section{Conversational Retrieval}

\section{Embeddings}

\section{Kaagle}

\section{Huggingface}

But the Hugging Face platform has the goal of exposing models, datasets, and docs about all ML models to maintain the ecosystem as open-source as possible, with the philosophy that the ML can evolve only with shared and open knowledge. The economic profits are huge, and products like chatGPT+, Bing (with a modified chatGPT), or Google Search (with Bard) don’t share the model to have advantages over the competitors.
