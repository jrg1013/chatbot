\capitulo{7}{Conclusiones y Líneas de trabajo futuras}

\section{El futuro de los LLM}

Aunque ChatGPT es la última novedad, no es más que un pequeño paso hacia lo que está por venir en el ámbito de los \acrshort{llm}. Aunque no se puede predecir el futuro, hay algunas tendencias que marcarán el camino de la innovación en los próximos anos~\cite{Tolaka}.

\begin{enumerate}
    
\item Modelos autónomos que se mejoran a sí mismos: Es probable que estos \acrshort{llm} tengan la capacidad de generar datos de entrenamiento para mejorar su propio rendimiento. Esto puede ser especialmente útil una vez agotadas las ingentes cantidades de información disponibles en Internet. Como ejemplo reciente, el \acrshort{llm} de Google fue capaz de generar sus propias preguntas y respuestas y ajustarse en consecuencia.

\item Modelos capaces de verificar sus propios resultados: Los \acrshort{llm} que pueden proporcionar fuentes para la información que generan pueden dar mayor credibilidad a la tecnología en su conjunto. Por ejemplo, WebGPT de OpenAI es capaz de generar respuestas precisas y detalladas con fuentes de respaldo.

\item El desarrollo de modelos expertos dispersos: Los \acrshort{llm} más reconocidos de la actualidad tienen varias características en común: son modelos densos, autosupervisados y preentrenados basados en la arquitectura de \textit{transformers}. Sin embargo, los modelos expertos dispersos están llevando la tecnología en otra dirección. Con estos modelos sólo es necesario activar los parámetros relevantes, lo que los hace más grandes y complejos. Al mismo tiempo, requieren menos recursos y consumo de energía para el entrenamiento del modelo.

\end{enumerate}

\section{Líneas de trabajo futuras}

El sector de la \acrlong{ia} y de los \acrlong{llm} está actualmente en constante evolución y ofrece enormes posibilidades. Existen multitud de posibles variaciones y evoluciones de \acrshort{tfg} para trabajar sobre con los \acrshort{llm}.

\begin{itemize}
    \item \textbf{Chatbot sin LangChain:} Existen multitud de artículos que apuntan a la falta de idoneidad de utilizar LangChain en producción~\cite{LangChainNot}. Como se ha mencionado anteriormente, LangChain está muy centrado en OpenAI, por lo que además no es el mejor \textit{framework} para \textit{Open Source} \acrshort{llm}. Se puede implementar una librería propia o probar alternativas con una capa de interacción distinta. Ambas opciones pueden ser interesantes y mostrarían mejor el funcionamiento de los \acrshort{llm} al tener que actuar con ellos directamente.

    \item \textbf{\acrshort{rag} on-demand:} Aunque ya existe este concepto y OpenAI lo ha mostrado como parte de su última presentación, sería interesante poder crear chatbots basados en datos que acabamos de proporcionar. Sería sencillamente, optimizar el proceso que se ha seguido en este \acrshort{tfg} para que se pueda realizar una base de datos vectorial en tiempo real y luego hacer preguntas con recuperación de datos de los documentos que se acaban de proveer.

    \item \textbf{Tracking de requisitos:} Un aspecto que cada vez tiene mas importancia en lo proyectos de desarrollo de software es la gestión de la calidad. Esto hace que se invierta mucho tiempo en determinar los requisitos y hacer pruebas que garanticen que estos requisitos han sido implementados de forma satisfactoria. Sin embargo esto conlleva que una parte nada despreciable de tiempo sea dedicada a una labor casi administrativa de comparación de documentos y seguimiento de los casos de prueba y sus resultados. Está gestión de requisitos a través de un \acrshort{llm} podría ayudar a automatizar la búsqueda de requisitos duplicados o inconsistentes, por ejemplo.

    \item \textbf{Miniaturización de \acrshort{llm}:} Uno de los grandes problemas encontrados en este \acrshort{tfg} ha sido los recursos que necesitan los \acrshort{llm} para ejecutarse. Esto hace que sea necesario muchas veces usar una \acrshort{api} para no tener un proceso demasiado lento o una instalación compleja. Probar Miniaturización de modelos para tareas muy especificas puede ser una aplicación fascinante que combine el poder los \acrshort{llm} aplicado a pequeños dispositivos.
\end{itemize}

\section{Reflexiones y conclusiones}