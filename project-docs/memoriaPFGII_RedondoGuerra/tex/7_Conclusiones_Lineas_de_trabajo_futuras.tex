\capitulo{7}{Conclusiones y Líneas de trabajo futuras}

Todo proyecto debe incluir las conclusiones que se derivan de su desarrollo. Éstas pueden ser de diferente índole, dependiendo de la tipología del proyecto, pero normalmente van a estar presentes un conjunto de conclusiones relacionadas con los resultados del proyecto y un conjunto de conclusiones técnicas. 
Además, resulta muy útil realizar un informe crítico indicando cómo se puede mejorar el proyecto, o cómo se puede continuar trabajando en la línea del proyecto realizado. 

\section{El futuro de los LLM }

Aunque ChatGPT es la última novedad, no es más que un pequeño paso hacia lo que está por venir en el ámbito de los \acrshort{llm}. Aunque no se puede predecir el futuro, hay algunas tendencias que marcarán el camino de la innovación en los próximos anos\cite{Tolaka}.

\begin{enumerate}
    
\item Modelos autónomos que se mejoran a sí mismos: Es probable que estos \acrshort{llm} tengan la capacidad de generar datos de entrenamiento para mejorar su propio rendimiento. Esto puede ser especialmente útil una vez agotadas las ingentes cantidades de información disponibles en Internet. Como ejemplo reciente, el \acrshort{llm} de Google fue capaz de generar sus propias preguntas y respuestas y ajustarse en consecuencia.

\item Modelos capaces de verificar sus propios resultados: Los \acrshort{llm} que pueden proporcionar fuentes para la información que generan pueden dar mayor credibilidad a la tecnología en su conjunto. Por ejemplo, WebGPT de OpenAI es capaz de generar respuestas precisas y detalladas con fuentes de respaldo.

\item El desarrollo de modelos expertos dispersos: Los \acrshort{llm} más reconocidos de la actualidad tienen varias características en común: son modelos densos, autosupervisados y preentrenados basados en la arquitectura de \textit{transformaers}. Sin embargo, los modelos expertos dispersos están llevando la tecnología en otra dirección. Con estos modelos sólo es necesario activar los parámetros relevantes, lo que los hace más grandes y complejos. Al mismo tiempo, requieren menos recursos y consumo de energía para el entrenamiento del modelo.

\end{enumerate}